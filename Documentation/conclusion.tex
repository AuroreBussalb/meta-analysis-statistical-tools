% adding the line below for Multifile document support with LatexTools Sublime package 
%!TEX root = manuscript.tex

% conclusion

\section{Conclusion}

This work offers an open source tool for running meta-analysis and \gls{saob}: the code and data used are available, 
assuring the transparency and replicability of these analysis. 

The results of the first part of the work confirm \citet{Cortese2016}'s findings in the light of recent published clinical work.
In particular, studies following a standard protocol as defined by \citet{Anrs2014} show significant improvements on \gls{pblind}
raters.

Besides this meta-analysis, we propose here a new method for tackling the heterogeneity of clinical data on \gls{nfb}. This method aims at identify factors 
as positively or negatively contributing to \gls{nfb} efficacy. 

Among the probable influencing factors, one of them is linked to the quality
of acquisition of the \gls{eeg}. This result indirectly confirms a mode of action through specific \gls{eeg} training.
Likewise, treatment intensity corroborates what is known of learning theory (memory consolidation) and indirectly validates it as
part of the mode of action. Eventually, our results show that therapeutic efficacy measured by teachers is reduced compared to parents.
However this result doesn't seem to be fully explained by the placebo effect. To estimate the placebo effect, it may be preferable to compare \gls{nfb}
to some biofeedback (for instance \gls{emg}-biofeedback). 

These elements converge to the conclusion that existing methodologies traditionally used to assess pills are not adapted to the evaluation 
of medical devices. Consequently, \gls{nfb} is probably efficacious and the risk/benefit ratio is certainly in favor of its use as the
clinical evidence stands today.
