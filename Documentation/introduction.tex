% adding the line below for Multifile document support with LatexTools Sublime package 
%!TEX root = manuscript.tex

% Introduction

\section{Introduction} 

\gls{adhd} is a common psychiatric disorder of childhood with an estimated prevalence of about 5\% in school-aged children yielding to an 
estimated 2.5 millions of children in Europe \citep{DSM-5}.

This neurodevelopmental disorder is characterized by impaired attention and/or hyperactivity/impulsivity, symptoms which may persist 
in adulthood with clinical significance which makes \gls{adhd} a life-long problem for many patients \citep{Faraone2006}. There is mounting evidence 
that objective neurophysiological  measurements also identify specific phenotypes: this was particularly reported with \gls{eeg} 
recordings \citep{loo2017}. In case of \gls{adhd}, more theta waves (4-7Hz) are present in the frontal area
 whereas there are less beta waves (12-32Hz) and \gls{smr} (13-15Hz) in the central area \citep{Monastra2005, Matouvsek1984,
 Janzen1995}. 

\gls{nfb} is a noninvasive technique based on behavioral therapy that aims to reduce the \gls{adhd} symptoms.
 The brain is trained to improve its own regulation by providing real-time video/audio information 
about its electrical activity measured from scalp electrodes \citep{Arns2015, Steffert2010}. In case of \gls{adhd}, several
 \gls{nfb} protocols have been proposed and investigated to decrease the symptoms: 
\begin{description}
	\item Protocols based on frequency band training: a child can be asked to enhance his \gls{smr} 
	while suppressing theta or beta \citep{Lubar1976}, or he can have to enhance beta
	while suppressing theta (this scenario is known as \gls{tbr}) \citep{Arns2013};
	\item Protocol based on the \gls{scp} training which consists in the regulation of cortical excitation 
	thresholds by focusing on activity generated by external cues 
	(similar to \gls{erps}) \citep{Heinrich2004, Banaschewski2007}; 
	\item Protocol based on \gls{erps} (P300) \citep{Fouillen2017}: \gls{adhd} children have a reduced P300 
	amplitude so it can be considered as a specific neurophysiological marker 
	of selective attention. 
\end{description} 

Thus, during \gls{nfb} sessions, the child has to concentrate in order to modify his waves: 
if he manages to change correctly his \gls{eeg} patterns, he will be rewarded thank to 
a positive visual or auditive feedback. The operant conditioning principle will enable the 
child to repeat more and more easily this task and thanks to the natural neuronal plasticity,
 a neuronal reorganization is observed \citep{VanDoren2017}. 

Since the early 1970s, \gls{nfb} has been investigating as a potential treatment for \gls{adhd} resulting 
in a large body of scientific literature \citep{Lubar1976, Rossiter1995, Linden1996, Maurizio2014}.
Besides, the efficacy of \gls{nfb} on the core symptoms of \gls{adhd} (inattention, hyperactivity and 
impulsivity) has been subject to several meta-analytic studies \citep{Loo2005, Lofthouse2012, Arns2009, Micoulaud2014, 
Sonuga-Barke2013}.   

The most recent meta-analysis solely on the efficacy of \gls{nfb} has been conducted by \citet{Cortese2016} in 
which 13 studies are included. Although only \gls{rcts} are selected, the authors of 
this meta-analysis have made some choices which have been debated by the community in particular by 
\citet{Micoulaud2016} who has criticized the use of a uncommon behavioral scale provided by \citet{Steiner2014}
 for the teachers' assessments and the inclusion of a pilot study carried out by \citet{Arnold2014} in the meta-analysis. 

Meta-analysis have a particularly important impact on the \gls{nfb} domain which is 
characterized by a clinical literature that is tremendously heterogeneous in terms 
of methods. Therefore, when a meta-analysis seems to present loopholes, it is important to
 make sure the choices made are wise and that the study was well conducted. 
While we mostly agree with \citet{Cortese2016}, we decided to replicate the methods he suggested
 to a few exceptions. Indeed, we have investigated the limitations
 of his work and amended as detailed below. So we replicated and updated this existing meta-analysis 
with new-found studies matching \citet{Cortese2016} inclusion criteria.  

This first step underlines the fact that performing a meta-analysis is complex because of the heterogeneity 
of the studies published on \gls{nfb}. Indeed, they differ on many points such as for instance trial
 methodology and \gls{nfb} implementation. So even if all included studies are conformed to an 
inclusion criteria, they remain different from each other. Since we supposed 
that the choices made by authors may lead to various \gls{nfb} results, we extended the replication 
and updating of \citet{Cortese2016} to a broader set of studies and used 
adequate statistical tools to take advantage of the heterogeneity of the methodological implementations 
(both clinical and technical) in order to identify which of the factors 
independently influences the reported \gls{es}. 

% number of words: 639





