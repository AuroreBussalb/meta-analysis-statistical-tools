% adding the line below for Multifile document support with LatexTools Sublime package 
%!TEX root = manuscript.tex

% 150 words (less than 150 required)

% Abstract

\begin{abstract}

\noindent Numerous trials and several meta-analysis have been published on the efficacy of Neurofeedback (NFB) applied to ADHD in children and adolescents. 
Because of the publication of new researches meeting the inclusion criteria of the latest meta-analysis on that topic, we decided to update it and to benchmark choices originally made by authors. 
Then, we extended the analysis with a novel method: the systematic analysis of biases (SAOB) that takes advantage of studies technical and methodological high heterogeneity 
rather than suffering from it. We supposed that the various choices made by authors in their clinical trials may lead to different levels of NFB efficacy. The update of the most recent meta-analysis with 
two new publications confirms the results originally obtained: effect sizes are significant when parents are the outcome (pvalue = 0.0017) unlike when teachers
(considered as probably blind) are (pvalue = 0.14). However, standard NFB protocols show significant improvements 
on probably blind raters (p-value = 0.043). The SAOB was performed on 31 studies and three different methods (a linear weighted, a linear allowing selection variable and a non-linear but
hierarchical) were applied. All these methods identify 3 elements that may have an impact on Neurofeedback efficacy. First, a more intensive treatment associates with higher efficacy. Second,
high-end EEG systems improve the effectiveness of NFB in ADHD. Eventually, the person assessing the evolution of the symptoms has an impact on results: teachers seems to see less improvement. 
Besides these results, we introduce here a new way to look into the heterogeneity of clinical trials.
\vskip 0.2in
\noindent keywords: ADHD, Neurofeedback, influencing factors, linear regression, decision tree, meta-analysis.

\end{abstract}

% frontiers : 350