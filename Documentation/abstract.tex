% adding the line below for Multifile document support with LatexTools Sublime package 
%!TEX root = manuscript.tex

% 148 words (less than 150 required)

% Abstract

\begin{abstract}

\noindent Neurofeedback is a noninvasive technique that aims to reduce the ADHD symptoms. Given the impact of meta-analysis on that subject,
we first propose to replicate and update the most recent one. Then, we try to identify factors with an influence on Neurofeedback 
based on the heterogeneity of studies using three multivariate approaches which associate factors with the within subject effect size.
The replication and update of the latest meta-analysis confirm the results obtained by its authors: effect sizes are not significant when 
probably blind ratings are the outcome whereas they are for most proximal raters. Analysis of factors identifies 3 elements which may have 
an impact on Neurofeedback efficacy: the length of the treatment, the quality of the acquisition of the signals and the person assessing 
the evolution of the symptoms. Besides these results, we introduce here a new way to look into the heterogeneity of clinical trials. 
\vskip 0.2in
\noindent keywords: ADHD, Neurofeedback, influencing factors, linear regression, decision tree, meta-analysis.

\end{abstract}

