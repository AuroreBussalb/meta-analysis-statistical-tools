% adding the line below for Multifile document support with LatexTools Sublime package 
%!TEX root = manuscript.tex

\section{Supplemental material}

\subsection{Perform a meta analysis}

To perform the meta-analysi several steps must be followed. First the choice of the model: this analysis is based on either one of the following 
statistical models \citep{Borenstein2009}:
\begin{description}
    \item \textit{The fixed-effect model}: the true \gls{es} (i.e the \gls{es} that would be observed with an infinitely 
		large sample size) is the same for all the studies in the analysis. The differences between the actually observed \gls{es}s 
		are due to sampling errors;
    \item \textit{The random-effects model}: the true \gls{es} could vary from study to study. The differences between the observed
		\gls{es}s are due to sampling errors but also to the various designs of the studies (for instance the number of participants or the implementation).
\end{description}

In the present case, although the studies included into the meta-analysis met the same criteria, they remained different from each other 
on various points, so the random effects model was more appropriate than the fixed-effect model.  

\subsubsection{Compute the effect size of each study}

First, the scores presented in the articles were extracted and the \gls{es} of each study as defined in \citet{Morris2008} 
was computed as in \cref{eq:metareview_effect_size}:

\begin{equation}
\label{eq:metareview_effect_size}
\text{ES} = c_p \left[ \frac{(M_{\text{post},T} - M_{\text{pre},T}) - (M_{\text{post},C} - M_{\text{pre},C}) }{\text{SD}_{\text{pre}}} \right ].
\end{equation} 
An \gls{es} is exactly equivalent to a z-score of a standard normal distribution, it is computed as mean pre- to post-treatment 
score change in the \gls{nfb} group ($M_{\text{pre},T}$, $M_{\text{post},T}$) minus the mean pre- to post- treatment score change 
in the control group ($M_{\text{pre},C}$, $M_{\text{post},C}$), divided by the pooled pretest standard deviation ($SD_{\text{pre}}$) 
defined as \cref{eq:stats_metareview_std_pre}:

\begin{equation}
\label{eq:stats_metareview_std_pre}
\text{SD}_{\text{pre}} = \sqrt{\frac{(n_T - 1)\text{SD}_{\text{pre},T}^2 + (n_C - 1)\text{SD}_{\text{pre},C}^2} {n_T + n_C - 2}},
\end{equation}
where $\text{SD}_{t,G}$ indicates the standard deviation for group $G$ at time $t$ and $n_G$ defines the sample size of each group; 
$c_p$ is a bias adjustment typically used for small sample sizes and defined as \cref{eq:metareview_correction_factor}:

\begin{equation}
\label{eq:metareview_correction_factor}
\text{cp} =  1 - \frac{3} {4(n_T + n_C - 2) - 1}. 
\end{equation} 

The means (first statistical moments) correspond to the mean average score over all scores given by raters to assess the \gls{adhd} symptoms. 
The standard deviations of the means correspond to the squared root of the second statistical moment, the variance. The variance measures how
 far a set of numbers are spread out from their average value. 

\subsubsection{Compute the variance of each effect size}

Then, the variance of each \gls{es} was computed as described in \cref{eq:metareview_variance_ES} \citep{Morris2008}:
\begin{equation}
\label{eq:metareview_variance_ES}
\sigma^2(\text{ES}) = c_p^2 \left (\frac{n_T + n_C - 2} {n_T + n_C - 4} \right ) \left  [ \frac{2(1-\rho)(n_T + n_C)} {n_Tn_C} + \text{ES}^2 \right ] - \text{ES}^2.
\end{equation} 

To compute the variance of the \gls{es}, the pooled within-group Pearson correlation $\rho$ (i.e the pre-post correlation) was required as described in \cref{eq:metareview_within_group_pearson_correlation} \citep{James2013}:

\begin{equation}
\label{eq:metareview_within_group_pearson_correlation}
\rho =  \frac{ \sum_{i=1}^{n} (\text{pre}_i - \mu_{\text{pre}})(\text{post}_i - \mu_{\text{post}}) } { \sqrt{ \sum_{i=1}^{n} (\text{pre}_i - \mu_{\text{pre}})^2} \sqrt{\sum_{i=1}^{n} (\text{post}_i - \mu_{\text{post}})^2} }, 
\end{equation}
where $n$ is the number of patients included in a study, $pre_i$, $post_i$ are score values for patient $i$ at pre- and post-test 
respectively, and $\mu_{pre}$, $\mu_{post}$ the mean scores over all patients. It is a measure of linear correlation between two variables. 
A value of 1 means that there is a positive correlation whereas a value of -1 means a negative correlation. When $\rho=0$, there is no
 linear correlation \citep{James2013}. In our case, this correlation was not known and the raw data were not available so we took an
 approximation: \citet{Balk2012} found that a value of 0.5 yields values closer to those computed with the right value of the correlation. 

Once variances were obtained with \cref{eq:metareview_variance_ES}, we could compute the standard error and the 95\% confidence interval of each \gls{es}. 

\subsubsection{Compute the weight of each study}

To compute the \gls{se} a weight must be assigned to each study. To obtain them several steps must be followed. At first, the fixed-effects model 
weight $w_{fixed}$ of each study $k$ was computed as defined in \citet{Borenstein2009} described in \cref{eq:metareview_weight_fixed_study}: 

\begin{equation}
\label{eq:metareview_weight_fixed_study}
w_{\text{fixed}_k} = \frac{1}{\sigma^2(\text{ES}_k)}.
\end{equation} 

Nevertheless, we chose to use the random effects model, so the weights associated to this model are different. To compute them, the between-studies 
variance $\tau^2$ is required. It was calculated in three steps described in \cref{eq:metareview_Q}, \cref{eq:metareview_C} and \cref{eq:metareview_Tau} 
\citep{Borenstein2009}:

\begin{equation}
\label{eq:metareview_Q}
Q = \sum_{k=1}^{K} (w_{\text{fixed}_k} \text{ES}_k^2),
\end{equation}

\begin{equation}
\label{eq:metareview_C}
C = \sum_{k=1}^{K} (w_{\text{fixed}_k} - \frac{ \sum_{k=1}^{K} (w_{\text{fixed}_k})^2 } { \sum_{k=1}^{K} (w_{\text{fixed}_k}) },
\end{equation}
with $K$ the total number of included studies.

\begin{equation}
\label{eq:metareview_Tau}
\tau^2 = \frac{Q - \text{df}}{C},
\end{equation}
with $\text{df} = K - 1$ the degrees of freedom.

The random-effects model takes into account the differences between the studies, so the weights are equal to the inverse of the addition between the 
within-study variance (the variance of the \gls{es}) and the between-studies variance as presented in \cref{eq:metareview_weight_study}:

\begin{equation}
\label{eq:metareview_weight_study}
w_k = \frac{1}{\sigma^2(\text{ES}_k) + \tau^2}.
\end{equation} 

\subsubsection{Compute the summary effect}

Eventually, the weighted average of the $K$ \gls{es} was computed to obtain the \gls{se} as described in 
\cref{eq:metareview_summary_effect} \citep{Borenstein2009}:

\begin{equation}
\label{eq:metareview_summary_effect}
\text{Se} = \frac{\sum_{k=1}^{K} w_k \text{ES}_k} {\sum_{k=1}^{K} w_k}.
\end{equation} 

Once the \gls{se}  is obtained, we can compute its variance, its standard error, its 95\% confidence interval, its p-value, 
and $I^2$ estimating effects size's between studies heterogeneity. 

\subsection{Scales used for replication}

\begin{table}[h!]
  \centering
  \caption{Clinical scales used to update \citet{Cortese2016} with our choices and the two new articles.}
% table clinical scales


\scriptsize
\begin{tabular}{l*{4}{c}r}
\centering
\textbf{Study} & \textbf{Outcome} & \textbf{Score Names - Parents ratings} & \textbf{Score Names - Teachers ratings} \\
\toprule
\multirow{3}{*}{ \cite{Arnold2014} } & Total & SNAP IV & SNAP IV \\
& Inattention & SNAP IV & SNAP IV \\
& Hyperactivity & SNAP IV & SNAP IV \\
\midrule
\multirow{3}{*} { \cite{Bakhshayesh2011} } & Total & German ADHD-RS & German ADHD-RS \\
& Inattention & German ADHD-RS & German ADHD-RS \\
& Hyperactivity & German ADHD-RS & German ADHD-RS \\
\midrule
\multirow{2}{*} { \cite{Baumeister2016} } & Total & DISYPS & - \\[2ex]
\midrule
\multirow{3}{*} { \cite{Beauregard2006} } & Total & CPRS & - \\
& Inattention & CPRS & - \\
& Hyperactivity & CPRS & - \\
\midrule
\multirow{3}{*} { \cite{Bink2014} } & Total & ADHD-RS self report & - \\
& Inattention & ADHD-RS self report & - \\
& Hyperactivity & ADHD-RS self report & - \\
\midrule
\multirow{2}{*} { \cite{Christiansen2014} } & Total & Conners-3 Parents & Conners-3 Teachers \\[2ex]
\midrule
\multirow{3}{*} { \cite{Gevensleben2009} } & Total & German ADHD-RS & German ADHD-RS \\
& Inattention & German ADHD-RS & German ADHD-RS \\
& Hyperactivity & German ADHD-RS & German ADHD-RS \\
\midrule
\multirow{1}{*} { \cite{Heinrich2004} } & Total & German ADHD-RS & - \\[2ex]
\midrule
\multirow{3}{*} { \cite{Holtmann2009} } & Total & German ADHD-RS & - \\
& Inattention & German ADHD-RS & - \\
& Hyperactivity & German ADHD-RS & - \\
\midrule
\multirow{2}{*} { \cite{Linden1996} } & Total & IOWA Conners & - \\
& Inattention & IOWA Conners & - \\
\midrule
\multirow{3}{*} { \cite{Maurizio2014} } & Total & CPRS & CTRS \\
& Inattention & CPRS & CTRS \\
& Hyperactivity & CPRS & CTRS \\
\midrule
\multirow{3}{*} { \cite{Steiner2011} } & Total & Conners Rating Scales Revised & Conners Rating Scales Revised \\
& Inattention & Conners Rating Scales Revised & Conners Rating Scales Revised \\
& Hyperactivity & Conners Rating Scales Revised & Conners Rating Scales Revised\\
\midrule
\multirow{3}{*} { \cite{Steiner2014} } & Total & Conners-3 Parents & Conners-3 Teachers \\
& Inattention & Conners-3 Parents & Conners-3 Teachers \\
& Hyperactivity & Conners-3 Parents & Conners-3 Teachers\\
\midrule
\multirow{3}{*} { \cite{Strehl2017} } & Total & German ADHD-RS & German ADHD-RS \\
& Inattention & German ADHD-RS  & German ADHD-RS \\
& Hyperactivity & German ADHD-RS & German ADHD-RS \\
\midrule
\multirow{3}{*} {\cite{VanDongen2013} } & Total & ADHD RS & ADHD RS \\
& Inattention & ADHD RS & ADHD RS \\
& Hyperactivity & ADHD RS & ADHD RS \\
\bottomrule
\end{tabular}
\footnotesize
\centering
SNAP: Wanson, Nolan and Pelham Questionnaire, ADHD-RS: ADHD Rating Scale, CPRS: Conners Parent Rating Scale, CTRS: Conners Teacher Rating Scale, BOSS Classroom Observation: Behavioral Observation of Students in Schools, DISYPS: Diagnostic System of Mental Disorders in Children and Adolescents





  \label{Table:Table_mr_clinical_scales_update_cortese}
\end{table}
