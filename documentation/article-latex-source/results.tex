% adding the line below for Multifile document support with LatexTools Sublime package 
%!TEX root = manuscript.tex


% Results

\section{Results}

\subsection{Selected studies}

Search terms entered in Pubmed returned \textcolor{red}{155} results during the last check on February 12, 2018, including 22 
articles used in previous meta-analyses on \gls{nfb}. Following the selection process illustrated 
in Figure~\ref{Figure:systematic_review_workflow}, \textcolor{red}{33} studies were included in the \gls{saob} and \textcolor{red}{16} in the meta-analysis, 
as summarized in Table~\ref{Table:table_factors_analysis_meta_analysis_list_studies}. The \textcolor{red}{33} studies selected for the \gls{saob} 
followed \citeauthor{Cortese2016}'s criteria, with the exception of the requirement for a control group. 
Indeed, since within-\gls{es} were considered in this analysis, a control group was not required.

\subsection{Meta-analysis}

The replication of \citeauthor{Cortese2016}'s results obtained are presented 
in Table~\ref{Table:meta_review_comparison_revman_and_python_with_choices}:

\begin{itemize}
    \item when computing the between-\gls{es} for \citet{Arnold2014} with the values after 40 sessions of \gls{nfb}, 
      smaller between-\gls{es} were found as compared to those found by \citet{Cortese2016}, which was unexpected since  
			the clinical efficacy is supposed to increase with the number of \gls{nfb} sessions. These lower between-\gls{es}
			impacted the \gls{se}: they were slightly lower when computed with this choice although they nonetheless remained significant (see the three first lines 
			of Table~\ref{Table:meta_review_comparison_revman_and_python_with_choices});  
    \item when relying on the teachers' ratings from the Conners-3 to compute the between-\gls{es} of \citet{Steiner2014}, 
		higher \glspl{se} were found in attention but not for total and hyperactivity score. However, this different choice of 
		scale did not affect the statistical significance of the \glspl{se} (see the three last lines 
			of Table~\ref{Table:meta_review_comparison_revman_and_python_with_choices}).
\end{itemize}

The meta-analysis was then extended by adding \textcolor{red}{three} new articles \citep{Strehl2017, Baumeister2016, Bazanova2018}. 
\citet{Bazanova2018} \textcolor{red}{gave parents' assessments for all outcomes}, \citet{Baumeister2016} provided results solely for parents' 
total outcome, whereas \citet{Strehl2017} gave both teachers' and parents' assessments for all outcomes. Despite favorable results 
for \gls{nfb}, particularly on parents' assessments, adding these \textcolor{red}{three} new studies did not change either the magnitude or the significance 
of the \gls{se}, for any outcome regardless of the raters, as illustrated in 
Figure~\ref{Figure:meta_review_forest_plots_update_meta_analysis_our_choices_no_colors_2-columns_fitting_image}. 

Regarding the "standard protocol" subgroup, \citet{Cortese2016} found all the outcomes significant except for the 
hyperactivity symptoms rated by teachers, which showed only a statistical trend (p-value = 0.11). Similar results 
were obtained when adding the most recent studies meeting this definition \citep{Strehl2017, Baumeister2016} (p-value = 0.11). 
The \gls{se} for the total outcome assessed by teachers remained significant with the addition of the new
\gls{rct} (p-value = 0.043), thus giving more strength to this result since it is now based on four studies including 283
patients in total.

As for the no-drug subgroup, \glspl{se} were found significant for the inattention symptoms assessed by parents (p-value = \textcolor{red}{0.017}). 
In addition, the differences in \citet{Arnold2014} values and the inclusion of \citet{Bazanova2018} caused a 
loss of significance in hyperactivity outcome for parents (p-value = \textcolor{red}{0.062}) compared to \citet{Cortese2016} 
(p-value = 0.016). \textcolor{red}{Only} \citet{Bazanova2018} \textcolor{red}{was included in this subgroup: in the two other 
studies the subjects were taking psychostimulants during the trial.}

All the clinical scales used to compute the between-\gls{es} following our choices are summarized in the Supplemental Materials.

\subsection{Factors influencing Neurofeedback}

This analysis was performed on \textcolor{red}{33} trials \textcolor{red}{(corresponding to 67 observations)} assessing the efficacy of \gls{nfb}, as presented 
in Table~\ref{Table:table_factors_analysis_meta_analysis_list_studies}. \textcolor{red}{The outlier rejection removed two training groups
of \citet{Bazanova2018} from the analysis because their within-\gls{es} were out of the bounds.} Among the \textcolor{red}{26} 
factors selected, \textcolor{red}{seven} were removed because there were too many missing observations or because they were too homogeneous: beta up in frontal areas, 
the use of a transfer card, the type of threshold for the discrete rewards (incremental or fixed), the \gls{eeg} quality equal of 3, the 
presence of a control group, \textcolor{red}{the individualization of the frequency bands based on the \gls{iapf}, and coupling 
\gls{nfb} with \gls{emg}-Biofeedback}.  

All results are presented in Table~\ref{Table:table_factors_analysis_results_summary}. These results, require 
careful interpretation since each technique provided slightly different results. These differences 
may depend on the different assumptions of the model and several other factors. Nonetheless, we are inclined to 
trust the findings that are consistent across methods. 

The \gls{wls} technique identified \textcolor{red}{nine} significant factors for an adjusted R-squared of \textcolor{red}{0.62} (see second column of 
Table~\ref{Table:table_factors_analysis_results_summary}). 
When applying the \gls{ols}, the same factors were significant \textcolor{red}{(except the \gls{eeg} quality equal of 2 and the presence of more than one active electrode)}
with a lower adjusted R-squared \textcolor{red}{(0.35)}. The \gls{lasso} regression selected \textcolor{red}{six} significant factors (see third column of 
Table~\ref{Table:table_factors_analysis_results_summary}). With these methods, a negative coefficient means 
that the factor is in favor of the efficacy of \gls{nfb}. The decision tree is presented in Figure~\ref{Figure:factors_analysis_decision_tree_results}: 
the best predictor in our case was the \gls{pblind} (see last column of 
Table~\ref{Table:table_factors_analysis_results_summary}). \textcolor{red}{Four} other factors also split the subsets; however, 
increasingly fewer samples are available the lower we get into the tree, making the interpretation increasingly doubtful.  

Several factors were common to the three methods used. These included, in particular: the assessment 
by a blind rater, the treatment length, and an \gls{eeg} quality score equal to 2 \textcolor{red}{(see lines 1, 9, and 19 of 
Table~\ref{Table:table_factors_analysis_results_summary})}.
The methods also agreed on the direction of the effect for these factors: 
a shorter treatment and recording of the \gls{eeg} with a good-quality system appears preferable, whereas teachers' assessment appears less favorable 
compared to parents' assessment.

\textcolor{red}{It is more doubtful the influence of the factors returned by only one or two methods (see lines 2, 3, 7, 8, 11, 16, 17, and 20
of Table~\ref{Table:table_factors_analysis_results_summary}). In particular: 
\begin{itemize}
\item both \gls{wls} and \gls{lasso} found that using more than one active electrode during \gls{nfb} appears to lead to an higher efficacy;
\item both \gls{wls} and the decision tree found that performing a higher number of sessions seems to be preferable;
\item both \gls{lasso} and the decision tree found that a higher number of sessions per week appear to positively influence the efficacy of the \gls{nfb} treatment.
\end{itemize}
Five factors were returned by only one of the methods: randomizing the groups, the \gls{irb} approval, the session length, the presence of a transfer phase, 
and the correction or rejection of ocular artifacts}

\textcolor{red}{Eight factors were never selected by the three methods (see lines 4, 5, 6, 12, 13, 14, 15 and 22 of 
Table~\ref{Table:table_factors_analysis_results_summary}): the children's minimum and maximum age, being on drugs during \gls{nfb} treatment,
the protocols \gls{smr}, beta up in central areas, theta down, and \gls{scp}, and the artifact correction based on amplitude.}
Thus, these factors overwhelmingly appear not to have an influence on \gls{nfb} efficacy. 

In the next section we discuss only the factors that were selected by at least two of the three methods. 

% words number = 1081
