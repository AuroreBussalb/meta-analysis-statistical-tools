% adding the line below for Multifile document support with LatexTools Sublime package 
%!TEX root = manuscript.tex


% Results

\section{Results}

\subsection{Studies selected}

Search terms entered in Pubmed returned 152 results during the last check on December 14, 2017\todo{maybe re-run search
  to update this date to the week before the latest recent match} including 28 articles included in previous
meta-analysis on \gls{nfb}. After the selection process illustrated in~\ref{Figure:systematic_review_workflow}, 
31 studies were included in the \gls{saob} and 15 in the meta-analysis as summarized in~\ref{Table:table_factors_analysis_meta_analysis_list_studies}.
The 31 studies selected for the \gls{saob} followed the \citeauthor{Cortese2016}'s criteria to the exception of the
requirement for a control group. 
Indeed, since within \gls{es} were considered in this analysis, we did not need a control group.

\subsection{Meta-analysis}

The validation of the Python module used for this work was successful (details available in Supplemental Materials) and the code was made
available online\cite{}.

The replication and the update of \citeauthor{Cortese2016}'s study was conducted by applying the choices described in the Materials and Methods part 
and the results obtained are presented in~\ref{Table:meta_review_comparison_revman_and_python_with_choices}:

\begin{itemize}
    \item when computing the \gls{es} for \citet{Arnold2014} with the values after 40 sessions of \gls{nfb}, we found
      smaller \gls{es} than \citet{Cortese2016}, which is someone counter intuitive as one expect the clinical efficacy
      to increase with the number of neurofeedback session;  
    \item when relying on the teachers' ratings from the Conners-3 to compute the \gls{es} of \citet{Steiner2014}, we found higher \gls{se} in attention but not 
      in total and hyperactivity. However, this different choice of scale did not affect the significance of the \glspl{se}.
\end{itemize}

The meta-analysisi was then extended by adding two new articles \citep{Strehl2017, Baumeister2016} found 
by applying the same search criteria and finding more recent matches. \citet{Baumeister2016} provided results only for parents total outcome whereas \citet{Strehl2017} gave teachers 
and parents' assessments for all outcomes. Despite favorable results for \gls{nfb}, particularly on parents' assessments, adding these two 
new studies did not change either the magnitude nor the significance of the \gls{se} for any outcome whatever the raters
as illustrated in~\ref{Figure:meta_review_forest_plots_update_meta_analysis_our_choices_no_colors_2-columns_fitting_image}. 
 
As initially suggested by \citeauthor{Cortese2016}, the analysis was ran on two subgroup of studies: one gathering studies following the standard protocol defined by \citet{Arns2014}
and a second only including participants without drug during the clinical trial. 

Regarding the `standard protocol' subgroup, \citet{Cortese2016} found all the outcomes significant except for the hyperactivity symptoms 
rated by teachers, which only showed a statistical trend (p-value = 0.11). Similar results was obtained when adding the most recent
studies meeting this definition \citep{Strehl2017}
(p-value = 0.11). The \gls{se} for the total outcome assessed by teachers remained significant with the addition of the two new
\glspl{rct} (p-value = 0.043) giving more weight to this result given that it now is based on 4 studies including ???
patients.\todo{check the numbers I aded here}

As for the no-drug subgroup, \glspl{se}\todo{se or es?} were found significant for the inattention symptoms assessed by parents (p-value = 0.013). 
Besides, the differences in \citet{Arnold2014} values caused a loss of significance in hyperactivity outcome for parents (p-value = 0.066)
compared to \citet{Cortese2016} (p-value = 0.016). The two new studies were not 
included in this subgroup because subjects were taking psychostimulants during the trial.

All the clinical scales used to compute the \gls{es} following our choices are summarized in the Supplemental Materials.

\subsection{Factors Influencing Neurofeedback}\todo{either capitalize in titles or not. Check guidelines, if american
  english, then titles should be capitilized - otherwise not. In any case, make it consistent.}

This analysis was performed on 31 trials assessing the efficacy of \gls{nfb} as presented in~\ref{Table:table_factors_analysis_meta_analysis_list_studies}. 
Among the 25 factors selected, 6 were removed for missing too many observations or beeing too homogeneous: beta up in frontal areas, 
the use of a transfer card, the type of threshold for the rewards (incremental or fixed), the EEG quality equal of 3
(\todo{describe briefly for more readability}) and the presence of a control group. 

The \gls{es} within subjects was computed for all available clinical scales of each study and then factors were associated with the computed \gls{es}
using three different methods: the \gls{wls}, the \gls{lasso}, and the decision tree. The global results were presented in~\ref{Table:table_factors_analysis_results_summary}.
These results highlight require a careful interpretation since each technique provides with slightly different
results. Understandably, these differences steered from the variations in hypothesis of each model. Yet, the more
methods identified a factor, the more confident one could be in its contribution to explain the \gls{es} under different
modeling hypothesis.
\todo[inline]{make sure that the latex file format is consistent througout, either by setting a common text-width or by
  adjusting wrapping (joins) throughout the manuscript.}

With the \gls{wls}, we\todo{Please update to style throuout to remove first person. I personally find it too personal
  here for instance, `The wls technique identified 8 factors contributing \ldots'} found that 8 factors were
significantly different from zero for an adjusted R-squared of 0.74.\todo{Comment briefly on the Rs value - is it good?
no?} 
When applying the \gls{ols}, the same factors were significant except the presence of a transfer phase, the protocol
theta down, and the artifact correction
based on amplitude with an adjusted R-squared of 0.42. The \gls{lasso} regression selected significant factors by setting to 0 the factors that did
not explain the model, here 12 factors contributed to the model. With these two methods, a negative coefficient meant that the factor was in favor of the \gls{nfb}.

Eventually, the decision tree presented in~\ref{Figure:factors_analysis_decision_tree_results} split the dataset
with a threshold value on the factor representing each node. From top to bottom, each factor is chosen to provide with a
split minimizing an \gls{mse} function. The best predictor is naturally the one at the top of the tree: in our case it was the \gls{pblind}. Five other factors also split the subsets,  
however, the lower we got into the tree, the less samples were available, making interpretation harder and harder.  

Several factors were common to the three methods we\todo{remove `we used' and preferably keep active form rather than pasive} used, in particular the treatment length, the assessment 
by a blind rater, and \gls{eeg} quality of 2 that were all three returned by the three methods\todo{again the eeg
  quality criteria is really not clear - you need to rephrase where ever you talk about it}. Besides, 
the methods agreed on the direction of the influence of these factors. So, a shorter treatment and recording the \gls{eeg} 
with a good system seemed preferable, whereas teachers' assessment appeared less favorable.

It was more difficult to interpret the influence of the factors returned by only one or two methods. Indeed, it was not clear if they were
selected due to an imprecision of the method or if they really had an impact on \gls{nfb} efficacy.
For instance, both \gls{wls} and \gls{lasso} found that relying on the amplitude of the signal to correct artifacts, and including a 
transfer phase seemed not to improve \gls{adhd} symptoms. Conversely, the \gls{irb} approval, a theta down protocol, and a high number 
of sessions per week appeared to positively influence the results. The decision tree and \gls{lasso} had in common the protocol \gls{smr}:
it was associated with lower \gls{es}. Five factors were returned by only one of the methods: the minimal age of the children, being on drugs 
during \gls{nfb} treatment, randomizing the groups and the \glspl{scp} protocol. To avoir over interpreting results we
only discuss below the factors that were selected by at least ??? of three methods. 

Eventually, five factors were selected by no method: the correction for the ocular artifact, the children maximum age, the number of sessions,
the protocol beta up in central areas and the presence of more than one active electrode. Thus, these factors appeared not to have an influence on
\gls{nfb} efficacy.   

% words number = 1032
