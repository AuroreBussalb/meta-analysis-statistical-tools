% adding the line below for Multifile document support with LatexTools Sublime package 
%!TEX root = manuscript.tex

% Abstract

\noindent Numerous trials and several meta-analysis have been published on the efficacy of Neurofeedback (NFB) applied
to Attention Deficit Hyperactivity Disorder (ADHD) in children and adolescents with inconsistent findings.

This work replicated the latest meta-analysis on that topic (2016) and, by doing so, benchmarked methodological choices
originally made and later challenged.

Furthermore, the meta-analysis was updated including two recently published randomized control trials.

This process revealed the heterogeneity of studies included in past meta-analysis, which questions the reliability of
their results.

The analysis was therefore completed with a novel method: the systematic analysis of biases (SAOB) that takes advantage
of studies technical and methodological heterogeneity rather than suffering from it.

The SAOB was performed on k = 31 studies meeting the same inclusion criteria as for the update of the meta-analysis (but the requirement for a control arm).

The update of the most recent meta-analysis with two new publications confirmed the results originally obtained: effect
sizes were significant when clinical scales of ADHD were rated by parents (p-value = 0.0017) but not when teachers did
(considered as probably blind, p-value = 0.14).

Also, significant improvements were confirmed for the subset of studies meeting the definition of "standard NFB
protocols" even when clinical outcomes were observed by probably blind raters (p-value = 0.043, k = 4 studies).

The SAOB identified 3 elements that might have an impact on NFB efficacy: first, a more intensive treatment was
associated with higher efficacy; second, high-end EEG systems improved the effectiveness of NFB in ADHD; third, the
person assessing the symptoms changes during trials had an impact on results: teachers seemed to score less improvement.

In conclusion, more than replicating previous findings, we introduced here a new way to look into the heterogeneity of
clinical trials.  

% frontiers: 350
