% adding the line below for Multifile document support with LatexTools Sublime
% package 
%!TEX root = manuscript.tex

% Abstract

\noindent Meta-analyses have been extensively used to
evaluate the efficacy of \gls{nfb} treatment for \gls{adhd} in children and adolescents. 
However, each meta-analysis published in the past decade has contradicted the methods and
results from the previous one, thus making it difficult to determine a consensus of opinion
on the effectiveness of \gls{nfb}. This works brings continuity to the field by extending and discussing the last and much 
controversial meta-analysis by \citet{Cortese2016}. 

The extension comprises an update of that work including the latest control trials, 
which have since been published and, most importantly, offers a
novel methodology. Specifically, \gls{nfb} literature is characterized 
by a high technical and methodological heterogeneity, which partly explains the current lack of consensus on 
the efficacy of \gls{nfb}. This work takes advantage of this by performing a \gls{saob} in studies included in the previous meta-analysis.

Our extended meta-analysis (k = \textcolor{red}{16} studies) confirmed the previously
obtained results of effect sizes in favor of \gls{nfb} efficacy as being significant when clinical scales of \gls{adhd}
are rated by parents (non-blind, p-value = \textcolor{red}{0.0014}), but not when they are rated by
teachers (probably blind, p-value = 0.27). The effect size is significant
according to both raters for the subset of studies meeting the definition of
"standard \gls{nfb} protocols" (parents' p-value = \textcolor{red}{0.0054}; teachers' p-value = 0.043, k
= 4). Following this, the \gls{saob} performed on k = \textcolor{red}{33} trials (meeting the same inclusion
criteria with the exception of the requirement of a control arm)
identified three main factors that have an impact on \gls{nfb} efficacy: first, a more
intensive treatment, but not treatment duration, is associated with higher
efficacy; second, teachers report a lower improvement compared to parents;
third, using high-quality EEG equipment improves the effectiveness of the \gls{nfb} treatment.

The identification of biases relating to an appropriate technical implementation of \gls{nfb} 
certainly supports the efficacy of \gls{nfb} as an intervention. The data presented also suggest that the \emph{probably blind} 
assessment of teachers may not be considered a good proxy for blind assessments, therefore stressing the need for 
studies with placebo-controlled intervention as well as carefully reported neuromarker changes in relation to clinical response.

% 348 words


