% adding the line below for Multifile document support with LatexTools Sublime package 
%!TEX root = manuscript.tex


% Abstract


\noindent Numerous trials and several meta-analysis have been published on the efficacy of Neurofeedback (NFB) applied to Attention Deficit Hyperactivity Disorder (ADHD) 
in children and adolescents with inconsistent findings. We decided to replicate the latest meta-analysis on that topic published in 2016, 
and by doing so, we benchmarked choices originally made by authors. Furthermore, we searched for new studies to add to the meta-analysis in order to update it.
This job highlighted the heterogeneity of studies included in meta-analysis, which may question the reliability of the results. We thus extended the analysis with a novel method: 
the systematic analysis of biases (SAOB) that took advantage of studies technical and methodological high heterogeneity rather than suffering from it.
The SAOB was performed on k = 31 studies and three different methods (a linear weighted, a linear allowing selection variable and a non-linear but hierarchical) were applied.
The update of the most recent meta-analysis with two new publications confirmed the results originally obtained: effect sizes were significant when parents were 
raters (p-value = 0.0017) unlike when teachers (considered as probably blind) were (p-value = 0.14). However, when only selecting studies fitting with standard NFB protocols definition,
significant improvements are observed on probably blind raters as well (p-value = 0.043, k = 4 studies). The SAOB identified 3 elements that might have an impact on NFB efficacy: 
first, a more intensive treatment was associated with higher efficacy, second, high-end EEG systems improved the effectiveness of NFB in ADHD, third, the person assessing the symptoms changes during 
trials had an impact on results: teachers seemed to score less improvement. In conclusion, more than replicating previous findings, we introduced here a new way to look 
into the heterogeneity of clinical trials.  

% frontiers : 350