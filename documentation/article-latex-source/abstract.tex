% adding the line below for Multifile document support with LatexTools Sublime
% package 
%!TEX root = manuscript.tex

% Abstract

\noindent Meta-analysis have been extensively used to
evaluate the efficacy of Neurofeedback (NFB) treatment for Attention
Deficit \comment[use gls] Hyperactivity Disorder (ADHD) in children and adolescents. 
Unfortunately, each meta-analysis published in the past decade has contradicted the methods and
results from the previous, making it particularly difficult to forge an opinion
on the effectiveness of NFB. 

This works brings continuity to the field by extending and discussing the last and much 
controversial meta-analysis by \citet{Cortese2016}. 

%The discussion covers methodological 
%choices that were criticized as well as the relevance of teacher assessments as \emph{probably blind}
%outcomes. 

The extension comprises an update of that work including the latest control trials 
(RCTs) that have been published since and, most importantly, a
novel methodology to look at this clinical literature. Specifically, NFB literature is characterized 
by a high technical and methodological heterogeneity, which partly explains the lack of consensus on 
the efficacy of NFB. This works takes advantage of it by performing a systematic analysis of
biases (SAOB) in studies included in the previous meta-analysis.

Our extended meta-analysis (k = 15 studies) confirmed the results previously
obtained: effect sizes in favor of NFB efficacy were significant when clinical scales of ADHD
were rated by parents (non-blind, p-value = 0.0017), but not when rated by
teachers (probably blind, p-value = 0.14). The effect size was significant
according to both raters for the subset of studies meeting the definition of
"standard NFB protocols" (parents p-value = 0.0054; teachers p-value = 0.043, k
= 4). Then, the SAOB, performed on k = 31 trials (meeting the same inclusion
criteria to the exception of the requirement of a control arm)
identified three main factors that have an impact on NFB efficacy: first, a more
intensive treatment, but not treatment duration, was associated with higher
efficacy; second, teachers reported a lower improvement as compared to parents;
third, high-quality EEG systems improved the effectiveness of the NFB treatment.

%This work reconciles the heterogeneity of previous meta-analysis results by
%suggesting it stems from a heterogeneous literature and offers a new method to
%look at such data revealing key contributing factors to clinical efficacy. 

The identification of biases relating to the good technical implementation of NFB 
certainly supports the efficacy of NFB as an
intervention. The data presented also suggests that teachers' \emph{probably blind} assessments, are
not a good proxy for blinding, 
%limiting the value of existing evidence 
and therefore stressing the need for studies with placebo-controlled
intervention as well as carefully reported neuromarker changes in relation to
clinical response. 

% frontiers: 350 
% 348 words


