% adding the line below for Multifile document support with LatexTools Sublime
% package 
%!TEX root = manuscript.tex

% Abstract

\textcolor{red}{\noindent Meta-analysis have been extensively used to
evaluate the efficacy of \gls{nfb} treatment for \gls{adhd} in children and adolescents. 
Unfortunately, each meta-analysis published in the past decade has contradicted the methods and
results from the previous, making it difficult to forge an opinion
on the effectiveness of \gls{nfb}. This works brings continuity to the field by extending and discussing the last and much 
controversial meta-analysis by} \citet{Cortese2016}. 

\textcolor{red}{The extension comprises an update of that work including the latest control trials 
that have been published since and, most importantly, a
novel methodology. Specifically, \gls{nfb} literature is characterized 
by a high technical and methodological heterogeneity, which partly explains the lack of consensus on 
the efficacy of \gls{nfb}. This works takes advantage of it by performing a \gls{saob} in studies included in the previous meta-analysis.}

Our extended meta-analysis (k = 15 studies) confirmed the results previously
obtained: effect sizes in favor of \gls{nfb} efficacy are significant when clinical scales of \gls{adhd}
are rated by parents (non-blind, p-value = 0.0017), but not when they are rated by
teachers (probably blind, p-value = 0.14). The effect size is significant
according to both raters for the subset of studies meeting the definition of
"standard \gls{nfb} protocols" (parents p-value = 0.0054; teachers p-value = 0.043, k
= 4). Then, the \gls{saob}, performed on k = 32 trials (meeting the same inclusion
criteria with the exception of the requirement of a control arm)
identified three main factors that have an impact on \gls{nfb} efficacy: first, a more
intensive treatment, but not treatment duration, is associated with higher
efficacy; second, teachers report a lower improvement as compared to parents;
third, using high-quality EEG equipment improves the effectiveness of the \gls{nfb} treatment.

\textcolor{red}{The identification of biases relating to an appropriate technical implementation of \gls{nfb} 
certainly supports the efficacy of \gls{nfb} as an
intervention. The data presented also suggests that the \emph{probably blind} assessment of teachers is
not a good proxy for blind assessments, therefore stressing the need for studies with placebo-controlled
intervention as well as carefully reported neuromarker changes in relation to
clinical response.}

% frontiers: 338 
% 348 words


