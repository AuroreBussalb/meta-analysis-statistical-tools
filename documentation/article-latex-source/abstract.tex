% adding the line below for Multifile document support with LatexTools Sublime package 
%!TEX root = manuscript.tex

% Abstract

\noindent Numerous trials and several meta-analysis have been published on the efficacy of 
the Neurofeedback (NFB) treatment for Attention Deficit Hyperactivity Disorder (ADHD) in children 
and adolescents, showing inconsistent findings. This work replicates and extents the last meta-analysis 
published on this subject \citep{Cortese2016} by adding two randomized control trials (RCTs) that have been published since.
We also perform a systematic analysis of biases (SAOB), which takes advantage
of the technical and methodological heterogeneity of the studies included in the meta-analyses
rather than suffering from it.
The SAOB was performed on k = 31 trials meeting the same inclusion criteria as the earlier meta-analysis 
(but the requirement for a control arm). Our extended meta-analysis confirmed the results previously obtained: effect
sizes were significant when clinical scales of ADHD were rated by parents (non-blind, p-value = 0.0017), but not when rated by 
teachers (considered as probably blind, p-value = 0.14). The effect size was significant according to both raters for 
the subset of studies meeting the definition of "standard NFB protocols" (parents p-value = 0.0054; teachers p-value = 0.043, 
k = 4 studies). The SAOB identified three main factors that might have an impact on NFB efficacy: first, a more intensive treatment, but
not treatment duration, was associated with higher efficacy; second, high-quality EEG systems improved the effectiveness of the NFB 
treatment; third, teachers reported a lower improvement as compared to parents. In addition, to all raw data we used (31 studies), 
we release a complete Python library for performing meta-analysis, for easing the replication of this and previous studies as well 
as for future projects. In conclusion, more than replicating previous findings, we introduced here a new way to look into the 
heterogeneity of clinical trials.

% frontiers: 350
