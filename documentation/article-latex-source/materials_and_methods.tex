% adding the line below for Multifile document support with LatexTools Sublime package 
%!TEX root = manuscript.tex

% Materials and Methods

\section{Materials and Methods}

\subsection{Studies selection}

Search terms described in Supplemental Materials \citep{Supplementalmaterial} were entered in Pubmed 
and studies included in previous meta-analysis were identified. Among these studies, those that 
satisfied each of these points were selected:
\begin{itemize}
	\item studies have to assess \gls{nfb} efficacy; 
	\item subjects must be diagnosed \gls{adhd} based on DSM-IV \citep{DSM-4}, DSM-V \citep{DSM-5}, 
	ICD-10 \citep{ICD101993} criteria or according to an expert psychiatrist; 
	\item studies have to be written in English, German or French;
	\item studies have to include at least 8 subjects in each group;
	\item subjects must be younger than 25 years old;
  \item publications (or subsequently their corresponding author) have to disclose sufficient data 
	to compute required metrics for the following analysis.
\end{itemize} 
The studies satisfying all these points were included in the \gls{saob}. To replicate and update 
\citeauthor{Cortese2016}'s meta-analysis, we then applied the inclusion criteria of this meta-analysis 
to the selected studies (the main difference being the presence of an active control). 

\subsection{Outcome definition} 

In included studies, the severity of \gls{adhd} symptoms have been assessed by parents and, when available, 
by teachers. \citet{Cortese2016} and \citet{Micoulaud2014} defined parents as \gls{mprox} raters who were 
not blind to the treatment of their child, as opposed to teachers who were considered as \gls{pblind} raters. 
This distinction is meant to assess the amplitude of the placebo effect where it is hypothesized that teachers 
who are presumed more blind to the intervention are less influenced in their assessment by their perception of it. 
Efficacy of \gls{nfb} was given for the following outcomes on clinical scales when available: inattention, 
hyperactivity/impulsivity, and total scores. The factors analysis was performed using the total score only.

\subsection{Meta-analysis}

Meta-analysis is typically used to aggregate results from different clinical investigations and offer a 
consolidated state of evidence. To aggregate results from different studies, it is necessary to assume 
some level of homogeneity in their design relative to the inclusion criteria, the specificity intervention, 
the presence and the type of control (active, semi-active, or non-active). Because studies occasionally 
use slight variations of a clinical scales and because of the clinical heterogeneity of patients and control, 
the scores are standardized before being pooled. The between \gls{es} is one of such standardized metrics, 
which we implemented in this paper as described in the Supplemental Materials \citep{Supplementalmaterial}. 

The work was carried with an open-source Python package developed for this work that offers a more transparent 
approach to the choice or parameters and increases replicability. This package was benchmark against 
RevMan v5.3 \citep{RevMan} by replicating \citet{Cortese2016}'s work and obtaining the same results. 
The code is made fully available on a GitHub repository \citep{Bussalb2018} including raw data for everyone 
to review its implementation, update it, or use it for their own projects. 
 
Before updating the \citet{Cortese2016}'s work with recently published studies fitting with their inclusion 
criteria \citep{Strehl2017, Baumeister2016}, we decided to run a sensitivity analysis investigating choices 
that later proved controversial \citep{Micoulaud2016}. Altogether, the changes investigated in our update 
included the following:
\begin{itemize}
\item the \gls{es} of \citeauthor{Arnold2014}'s study was computed from the post-test clinical values taken 
after the completion of the 40 sessions at the opposite of \citet{Cortese2016}'s report which used the results 
after only 12 sessions because the end point values were not available at the time of his study;
\item the \gls{es} computed from teachers' assessments reported by \citet{Steiner2014} relied on the BOSS 
Classroom Observation \citep{Shapiro2010}. This was an atypical scale to quantify \gls{adhd} symptoms since 
the Conners Rating Scale Revised \citep{Conners1998} \citep{Christiansen2014, Bluschke2016}, whose metrics is 
well-defined \citep{Collett2003, Epstein2012}, was more often used and available in this study. Thus, we decided 
to compute the \gls{es} based on the results from the Conners-3, which was used in this study to compute the 
\gls{es} for the parents' ratings.  
\end{itemize} 

As initially suggested, we also performed two subgroups analysis with the two choices described above: 
first, only  with studies following standard protocol as defined by \citet{Arns2014} and then with studies 
whose participants took low-dose or no medication during the trial.  

\subsection{Identify factors influencing the Neurofeedback}

While revisiting the meta-analysis, it became apparent that the studies pooled together where highly heterogeneous 
in terms of methodological and practical implementation as pointed out by \citet{Alkoby2017}. For instance, all \gls{nfb} 
interventions were pooled together irrelevant to the quality of the acquisition, the level excerted on real time data 
quality, and the trained neuromarker. Likewise, the methodological implementations varied significantly, requiring the 
'subgroup' analysis (low drug, standard protocols) that are somewhat arbitrary. To circumvent these limitations, we 
implemented a novel approach: the \gls{saob}. In this setting, the within-\gls{es} of each intervention was considered 
as a dependent variable that methodological and technical biases explained using standard statistical tools. The results 
of such analysis should enable us to identify known methodological biases (e.g.\ blind assessments negatively associated 
with \gls{es}) and possibly technical factors (e.g.\ a good control on real time data quality influences positively the 
treatment outcome). 

\subsubsection{Identify and pre-process factors}

We classified the factors influencing the efficacy of \gls{nfb} in 5 categories: methodological, technical,
demographics, and quality of the signal acquisition. 
Factors were chosen based on what was typically reported in the literature, presumed to influence \gls{es}, 
and categorized as follow:

\begin{itemize}
  \item \emph{the signal quality}: correction of ocular and generic (amplitude based) artifacts correction;
  \item \emph{the population}: the psychostimulants intake during \gls{nfb} treatment and the age bounds of children
  included;
  \item \emph{the methodological biases}: the presence of a control group, the blinding of assessors, 
  the randomization of subjects, and the approval of the study by an \gls{irb};
  \item \emph{the \gls{nfb} implementation}: the protocol used (\gls{scp}, \gls{smr}, 
  theta up, beta up in central areas, theta down), the presence of a transfer phase during \gls{nfb} training, the 
	possibility to train at home or at school with a transfer card reminding of the \gls{nfb} session, 
  the type of thresholding reward, the number of \gls{nfb} sessions, the sessions frequency, and the sessions and
  treatment lengths;
  \item \emph{the acquisition quality}: the presence of one or more active electrodes and the \gls{eeg} quality. 
  This last factor was coded as an indicator between 1 and 3. To get an indicator equal to three, the following three  
  points had to be satisfied: 
	\begin{description}
	  \item[\emph{the type of electrodes used}]: \gls{agcl}/Gel and \gls{au}/Gel;
    \item[\emph{the use of impedance mode}]: a check of electrode contacts quality trough the amplifier impedance
     acquisition mode so that inter-electrode impedance are smaller than $40k\Omega$;  
    \item[\emph{the level of hardware certificate}]: compliance with ISO-60601-1-26\cite{} including the following devices 
		\citet{TheraPrax} (NeuroConn\copyright, Munich, Germany) and \citet{Eemagine} (ANT Neuro, Berlin, Germany) were 
		preferable.
	\end{description}
\end{itemize}	
If at least one point was checked, the quality was set to 2, otherwise a score of 1 was given for this criteria.

We provided in the Github repository \citep{Bussalb2018} the raw data extracted from the publications. To prevent any
bias in the analysis, the names of the factors were hidden during the entire analysis so that the data scientist (AB) 
was fully blind to them. The variable names were only revealed once the data  analysis and results were accepted as valid: 
this included choice of variable normalization and validation of model hypothesis as detailed below.

The pre-processing of factors for the analysis included the following steps: factors for which there were too many 
missing observations, arbitrarily set to more than 20\% of the total of observations, were removed from the analysis. 
Furthermore, if a factor had more than 80\% similar observations it was removed as well. Categorical variables were 
coded as dummies meaning that the presence of the factor was represented with 1 and its absence 0. All variables 
were standardized by subtracting the mean and then dividing this result by the standard deviation, except when 
the decision tree was performed. 

\subsubsection{Explaining effect sizes with factors}

To compute the \gls{es}, the  means of total \gls{adhd} scores given by parents and teachers were used. Besides, 
in case studies provided results for more than one behavioral scale, \gls{es} were computed for each one as 

\begin{equation*}
\label{eq:factors_effect_size_within_subject}
\text{ES} = \frac{M_{\text{post},T} - M_{\text{pre},T}}{\sqrt{\frac{\sigma_{\text{pre},T}^2 + \sigma_{\text{post},T}^2}{2}}},
\end{equation*} 

where $M_{t,T}$ is the mean of clinical scale, for treatment $T$, taken at time $t$ (pre-test or post-test) and $\sigma$ represents 
its standard deviation.

The \gls{es} computed in this analysis was different from the one used for the replication and update of \citet{Cortese2016}. 
Indeed, here we focused on the effect of the treatment within a group as defined by \citet{Cohen1988} using a definition of 
the \gls{es} that was already used in the literature \citep{Arns2009, Maurizio2014, Strehl2017}. This \gls{es} enables to quantify 
the efficacy of \gls{nfb} inside the treatment group. 

The \gls{es} was then considered as a dependent variable to be explained by the factors (the independent variables). 
The following three methods, implemented in the Scikit-Learn Python \citep[0.18.1]{Pedregosa2011} and the Statsmodels Python
\citep[0.8.0]{Seabold2010} libraries, were used to perform the regression:
\begin{itemize}
\item weighted multiple linear regression (\gls{wls}) \citep{Montgomery2012};
	\item sparsity-regularized linear regression with \gls{lasso} \citep{Tibshirani1996};
	\item decision tree \citep{Quinlan1986}.
\end{itemize}

To illustrate simply our approach, the aim of the linear regression was first to estimate the regression coefficients
linking the factors to the \gls{es}. A significant coefficient (meaning significantly different from 0) indicated 
that the associated factor had probably an influence on \gls{nfb} efficacy and the sign of the coefficient indicated 
the direction of the effect.

The \gls{wls} varies from a traditional linear regression estimated with \gls{ols} in the sense that a weight was assigned 
to each observation so as to account for the multiplicity of reported clinical endpoints in some studies. Besides, the 
weight was also a function of the sample size to account for variations in sample sizes. Consequently, the weight of each study 
was taken as the ratio between the experiment group's sample size and the number of behavioral scales available.
We also ran the analysis with \gls{ols} method to assess the impact of the weights on the results. 

The second linear method applied was the \gls{lasso}, that naturally incorporates variable selection 
to the linear model thanks to $\ell-1$-norm applied on the coefficients. A coefficient not set at zero means that 
the associated factor may have had an influence on \gls{nfb} and, once again, the sign of the coefficient indicated the 
direction of the effect.

The last method used to determine factors influencing \gls{nfb} was the decision tree \citep{Quinlan1986}, a hierarchical 
and non linear method. It braked down a dataset into smaller and smaller subsets using, at each iteration, a variable and 
a threshold chosen to optimize a simple \gls{mse} criteria \citep{James2013}. A tree is composed of several nodes and leafs, 
the importance of which is decreasing from the top node called root node. 

These methods are intrinsically different from each others, so we compared their results. For instance, the decision
tree captures variable interactions and can relate factors to \gls{es} in a non-linear fashion. On the other hand, the
\gls{lasso} offers an elegant mathematical framework to variable selection. For more details, please refer to the Supplemental Material
\citep{Supplementalmaterial}.


%1661













