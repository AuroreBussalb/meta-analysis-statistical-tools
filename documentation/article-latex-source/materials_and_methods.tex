% adding the line below for Multifile document support with LatexTools Sublime package 
%!TEX root = manuscript.tex

% Materials and Methods

\section{Materials and Methods}

\subsection{Studies selection}

Search terms described in Supplemental Material were entered in Pubmed and studies included in previous meta-analysis were identified. Among these studies, those that
satisfied each of these points were selected:
\begin{itemize}
	\item studies have to assess \gls{nfb} efficacy; 
	\item subjects must be diagnosed \gls{adhd} based on DSM-IV \citep{DSM-4}, DSM-V \citep{DSM-5}, ICD-10 \citep{ICD101993} 
	criteria or according to an expert psychiatrist; 
	\item studies have to be written in English, German or French;
	\item studies have to include at least 8 subjects in each group;
	\item subjects must be younger than 25 years old;
	\item studies have to provide enough data to compute required metrics for the following analysis.
\end{itemize} 
The studies satisfying all these points were included in the \gls{saob}. To replicate and update \citeauthor{Cortese2016}'s meta-analysis, we then applied the inclusion 
criteria of this meta-analysis to the selected studies. 

\subsection{Outcome definition} 

In included studies, the severity of \gls{adhd} symptoms have been assessed by parents and, when available, by teachers. \citet{Cortese2016} 
and \citet{Micoulaud2014} defined parents as \gls{mprox} raters who were not blind to the treatment of their child, as opposed to 
teachers who were considered as \gls{pblind} raters. This distinction is meant to assess the amplitude of the placebo effect where 
it is hypothesized that teachers who are presumed more blind to the intervention are less influenced in their assessment by their perception of it. 
Efficacy of \gls{nfb} was given for the following outcomes on clinical scales when available: inattention, 
hyperactivity/impulsivity and total scores. The factors analysis was performed using the total score only.

\subsection{Meta-analysis}

Meta-analysis is typically used to aggregate results from different clinical investigations and offer a consolidated 
state of evidence. To aggregate results from different studies, it is necessary to assume some level of homogeneity 
in their design relative to the inclusion criteria, the specificity intervention, the presence and the type of control (active, semi-active or 
non-active). Because studies occasionally use slight variations of a clinical scales and because of the clinical heterogeneity of patients and control, 
the scores are standardized before being pooled. 
The between \gls{es} is one of such standardized metrics, which we implemented in this paper as described 
in the Supplemental Materials. The work was carried with an open-source 
Python package developed for this work that offers a more transparent approach to the choice or parameters 
and increases replicability. This package was benchmark against RevMan v5.3 \citep{RevMan}
by replicating \citet{Cortese2016}'s work and obtaining the same results. The code is made fully available 
on a GitHub repository \citep{Bussalb2018} including raw data for everyone to review its implementation, update it, or 
use it for their own projects. 
 
Before updating the \citet{Cortese2016}'s work with recently published studies fitting with their inclusion criteria 
\citep{Strehl2017, Baumeister2016}, we decided to run a sensitivity analysis investigating choices that later 
proved controversial \citep{Micoulaud2016}. Altogether, the changes investigated in our update included the following:
\begin{itemize}
\item the \gls{es} of \citeauthor{Arnold2014}'s study was computed from the post-test clinical values taken after the completion of the 40 sessions 
at the opposite of \citet{Cortese2016}'s report which used the results after 12 sessions of \gls{nfb} because the end point values were not available;
\item the \gls{es} computed from teachers' assessments reported by \citet{Steiner2014} relied on the BOSS Classroom Observation \citep{Shapiro2010}. This
was an atypical scale to quantify \gls{adhd} symptoms since the ADHD-Rating scale \citep{Pappas2006} or the Conners Rating Scale Revised \citep{Conners1998}
\citep{Christiansen2014, Bluschke2016} whose metrics were well-defined \citep{Collett2003, Epstein2012}, were more often used. Thus, we decided to compute
the \gls{es} based on the results from the Conners-3, which was used in this study to compute the \gls{es} for the parents' ratings.  
\end{itemize} 

As initially suggested, we performed two subgroups analysis with the two choices described above: the \gls{se}, the weighted average of all the
\gls{es}, was calculated first with only studies following standard protocol as defined by \citet{Arns2014} and in a second time with studies 
whose participants took low-dose or no medication during the trial.  

\subsection{Detect factors influencing the Neurofeedback}

While revisiting the work carried on meta-analysis, it became apparent that the studies pooled together where highly heterogeneous 
in terms of methodological and practical implementation as pointed out by \citet{Alkoby2017}. For instance, all \gls{nfb} interventions were pooled together irrelevant to the 
quality of the acquisition, the level excreted on real time data quality, and the trained neuromarker. 
Likewise, the methodological implementations varied significantly, requiring the 'subgroup' analysis (low drug, standard protocols) 
that are somewhat arbitrary. To circumvent these limitations, we implemented a novel approach: the \gls{saob}. 
In this setting, the within-\gls{es} of each intervention was considered as a dependent variable that
methodological and technical biases tried to explain using standard statistical tools. The results of such analysis should enable us to identify 
known methodological biases (e.g.\ blind assessments negatively associated with \gls{es}) and possibly technical factors (e.g.\ a good control on 
real time data quality influences positively the treatment outcome). 

\subsubsection{Identify and pre-process factors}

We classified the factors influencing the efficacy of \gls{nfb} in 5 categories: methodological, technical, characteristics of the included
populations, representative of the quality of acquisition and of the signal. 
Factors were chosen based on what was typically reported in the literature and presumed to influence \gls{es} and categorized as follow:

\begin{itemize}
\item \emph{the signal quality}: correction for the presence of ocular artifacts and correction for artifacts based on amplitude; 
\item \emph{the population}: the psychostimulants intake during \gls{nfb} treatment and the age bounds of children;
\item \emph{the methodological biases}: the presence of a control group, the blinding of assessors, 
the randomization of subjects, and the approval of the study by an \gls{irb};
\item \emph{the \gls{nfb} implementation}: the protocol used (\gls{scp}, \gls{smr}, 
theta up, beta up in central areas, theta down), the presence of a transfer phase during \gls{nfb} training, the possibility to train at home 
or at school with a transfer card reminding of the \gls{nfb} session, 
the type of thresholding reward, the number of \gls{nfb} sessions, the sessions frequency during a week, the session length and the treatment length;
\item \emph{the acquisition quality}: the presence of one or more active electrodes and the \gls{eeg} quality. 
This last factor was an indicator between 1 and 3: if \gls{eeg} was recorded and processed in poor conditions then the indicator was 1. 
Besides, if the article did not mention the recording conditions, the factor was set to 1. To get an indicator bigger than 1, several 
points had to be satisfied:
\begin{itemize}
  \item \emph{the type of electrodes used}: \gls{agcl}/Gel and \gls{au}/Gel;
  \item \emph{a check of electrode contacts quality trough the amplifier impedance acquisition mode}: inter-electrode impedance should have to be smaller than $40k\Omega$;  
  \item \emph{the type of amplifier used}: those that were in accordance with European standards (such as Thera Prax\textregistered, by NeuroConn\copyright,
	Germany \href{https://www.neurocaregroup.com/neuroconn-thera-prax.html}{NeuroCare} and Eemagine\copyright, Germany \href{http://www.eemagine.com/}{Eemagine}) were preferable or 
	whose reliability is known.
\end{itemize}
If at least one point was checked, the quality was set to 2. To get an indicator of 3, all these criteria had to be satisfied.
\end{itemize}

We provided in the Github repository \citep{Bussalb2018} the raw data extracted from the publications. To prevent any bias, the names of these factors
were hidden during the entire analysis so that the data scientist (AB) was fully blind to them. The variable names were only revealed once the data 
analysis and results were accepted as valid: this included choice of variable normalization and validation of model hypothesis as detailed below.

The pre-processing of factors for the analysis included the following steps: factors for which there were too many missing observations, 
arbitrarily set to more than 20\% of the total of observations, were removed from the analysis. Furthermore, if a factor had more than 
80\% similar observations it was removed as well. Categorical variables were coded as dummies meaning that the presence of the factor was represented by a 1 
and its absence by 0. All variables were standardized, except when the decision tree was performed. 

\subsubsection{Associate independent factors to effect sizes}

To compute this \gls{es}, means of total \gls{adhd} scores given by parents and teachers were used. Besides, in case studies provided results 
for more than one behavioral scale, \gls{es} were computed for each one as 

\begin{equation*}
\label{eq:factors_effect_size_within_subject}
\text{ES} = \frac{M_{\text{post},T} - M_{\text{pre},T}}{\sqrt{\frac{\sigma_{\text{pre},T}^2 + \sigma_{\text{post},T}^2}{2}}},
\end{equation*} 

where $M_{t,T}$ is the mean of clinical scale taken at time $t$ (pre-test or post-test) and for treatment $T$ and $\sigma$ similarly represents its standard deviation.

The \gls{es} computed in this analysis was different from the one 
used previously for the replication and updating of \citet{Cortese2016}. Indeed, here we focused on the effect of the treatment within 
a group as defined by \citet{Cohen1988}, definition of the \gls{es} that was already used in the literature \citep{Arns2009, Maurizio2014, 
Strehl2017}. This \gls{es} enables to quantify the efficacy of \gls{nfb} inside the treatment group 

The \gls{es} was then considered as a dependent variable to be explained by the factors (the independent variables) identified using the following three methods, which were 
implemented in the Scikit-Learn Python \citep[0.18.1]{Pedregosa2011} and the Statsmodels Python \citep[0.8.0]{Seabold2010} libraries:
\begin{itemize}
	\item weighted multiple linear regression (\gls{wls}) \citep{Montgomery2012}; 
	\item sparsity-regularized linear regression with \gls{lasso} \citep{Tibshirani1996};
	\item decision tree \citep{Quinlan1986}.
\end{itemize}

The most often used linear regression analysis is the \gls{ols} but here we applied the \gls{wls}: a 
weight was assigned to each observation in order to take into account the fact that some observations came from the same study because studies 
may provide several scales. Besides, the weight was a function of the sample size as well: due to their different sample sizes,
studies were not equivalent and should be analyzed accordingly. That's why the weight corresponded to the ratio between the experiment group's sample size of the study and 
the number of behavioral scales available in the study. We also ran the analysis with \gls{ols} method to assess the impact of the weights on the results. 

The aim of the \gls{wls} was to estimate the regression coefficients. A significant coefficient (meaning significantly different from 0) indicated 
that the associated factor had probably an influence on \gls{nfb} efficacy and the sign of the coefficient indicated the direction of the effect.

The second linear method applied was the \gls{lasso}, that naturally incorporates variable selection 
in the linear model thanks to $\ell-1$-norm applied on the coefficients. A coefficient not set at zero means that 
the associated factor may have had an influence on \gls{nfb} and once again, the sign of the coefficient indicated the direction of the effect.

Eventually, the last method used to determine factors influencing \gls{nfb} was the decision tree \citep{Quinlan1986}, a hierarchical and non linear method.
It braked down a dataset into smaller and smaller subsets using at each iteration a variable and a threshold chosen to optimize a simple \gls{mse} 
criteria \citep{James2013}. A tree is composed of several nodes and leafs, the importance of which is decreasing from the top node called root node. 

These methods are intrinsically different from each others, so we compared their results. They are more precisely described in the Supplemental Material.


%1661













