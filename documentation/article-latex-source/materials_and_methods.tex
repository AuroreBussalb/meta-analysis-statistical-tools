% adding the line below for Multifile document support with LatexTools Sublime package 
%!TEX root = manuscript.tex

% Materials and Methods

\section{Materials and Methods}

\subsection{Inclusion criteria}

Search terms were directly taken from \citet{Cortese2016} with the exception of the need for a control arm, 
which is detailed in Supplemental Materials \citep{Supplementalmaterial}. The requirements included:
\begin{itemize}
	\item studies have to assess \gls{nfb} efficacy; 
	\item subjects must have received a diagnosis of \gls{adhd} based on DSM-IV \citep{DSM-4}, DSM-5 \citep{DSM-5}, 
	ICD-10 \citep{ICD101993} criteria, or by a qualified psychiatrist; 
	\item studies have to be written in English, German, Spanish, or French;
	\item studies have to include at least eight subjects in each group;
	\item patients must be younger than 25 years old;
  \item the publication has to disclose sufficient details about the data
	to compute required metrics for the ensuing analysis.
\end{itemize} 
The studies satisfying all these criteria were included in the \gls{saob}.
In order to replicate and update \citeauthor{Cortese2016}'s meta-analysis, we applied the original inclusion criteria of 
their meta-analysis to our search (the main difference being the presence of a control group). 

\subsection{Outcome definition} 

In the included studies, the severity of \gls{adhd} symptoms have been assessed by parents and, whenever available, 
by teachers. \citet{Cortese2016} and \citet{Micoulaud2014} defined parents as \gls{mprox} raters who were 
not blind to the treatment, as opposed to teachers, who were considered as \gls{pblind} raters. 
This distinction is intended to assess the amplitude of the placebo effect, where it is hypothesized that teachers, 
who are presumed to be more blind to the intervention, are less influenced in their assessment. 
Efficacy of \gls{nfb} was measured using clinical scales, such as the \gls{adhd}-RS \citep{Pappas2006}, 
on the following outcomes: inattention, hyperactivity/impulsivity, and total scores. The factor analysis was 
performed using the total score.

\subsection{Meta-analysis}

The goal of a meta-analysis is to aggregate results from different clinical investigations and offer a 
consolidated body of evidence. To achieve this, it is necessary to assume 
some level of homogeneity in the design of the studies: inclusion criteria, and the presence and type of control 
(active, semi-active, or non-active). Because studies occasionally 
use slight variations of a clinical scale and because of the clinical heterogeneity of patients and control, 
the scores are standardized before being pooled into a \gls{se}. The between-\gls{es} is one such standardized metric, 
which we have implemented in this paper (see Supplemental Materials \citep{Supplementalmaterial}). 

The meta-analysis was performed with a Python package developed for this work. The package offers a transparent 
approach for the choice of parameters in an effort to ease replicability. We have benchmarked it against RevMan version 5.1 
\citep[UK, London]{RevMan} by replicating \citet{Cortese2016}'s work. 
The code is made fully available on a GitHub repository \citep{Bussalb2019}, together with all the \glspl{rct} raw data we have used
in the present study.
 
Before updating the \citet{Cortese2016} work with recently published studies
\citep{Strehl2017, Baumeister2016}, we decided to run a sensitivity analysis investigating the choices 
that later proved controversial \citep{Micoulaud2016}. The investigated changes included:
\begin{itemize}
\item the between-\gls{es} of \citeauthor{Arnold2014}'s study was computed from the post-test clinical values taken 
after the completion of the 40 sessions, in contrast to \citet{Cortese2016}'s report which used the results 
after only 12 sessions because the endpoint values were not available at the time of his study;
\item the between-\gls{es} computed from the teachers' assessment reported by \citet{Steiner2014} relied on the BOSS 
Classroom Observation \citep{Shapiro2010}. This is an atypical scale to quantify \gls{adhd} symptoms since 
the Conners Rating Scale Revised \citep{Conners1998, Christiansen2014, Bluschke2016}, a well-defined
\citep{Collett2003, Epstein2012} and broadly used metric, was available in this study. Thus, we decided 
to compute the between-\gls{es} based on the Conners-3 already used in this study to compute the 
parents' between-\gls{es}.  
\end{itemize} 

As initially suggested by \citeauthor{Cortese2016}, the analysis was run on two subgroups of studies 
with the two choices described above: one gathering studies following the standard protocol defined by 
\citet{Arns2014} and a second including only participants not taking medications during the clinical trial.

\subsection{Identify factors influencing the neurofeedback}

While revisiting the existing meta-analyses, it became apparent that the studies pooled together were highly heterogeneous 
in terms of methodological and practical implementation. For instance, all \gls{nfb} 
interventions were pooled together regardless the quality of acquisition, the quality of EEG data, and the trained 
neuromarker. Equally, the methodological implementations varied significantly, requiring the 
'subgroup' analysis (for instance, gathering studies following standard protocols) that are somewhat arbitrary. To circumvent these limitations, we 
implemented a novel approach: the \gls{saob}. With this method, the within-\gls{es} of each intervention was considered 
as a dependent variable to be explained by methodological and technical factors. Such analysis aims at identifying known methodological 
biases (e.g.\ blind assessments negatively associated with within-\gls{es}) and possible technical factors (e.g.\ a good control on real time 
data quality positively influences the treatment outcome). 

\subsubsection{Identify and pre-process factors}

We classified the factors influencing the efficacy of \gls{nfb} into five categories: methodological, technical,
demographics, and quality of the signal and acquisition. 
Factors were chosen based on that reported in the literature as presumed to influence \gls{es}, and categorized as follows:

\begin{itemize}
  \item \emph{the methodological biases}: the presence of a control group, the blindness of assessors, 
  the randomization of subjects in controlled trials, and the approval of the study by an \gls{irb};
  \item \emph{the population}: intake of psychostimulants during \gls{nfb} treatment, the age range of children
  included\textcolor{red}{, the severity of \gls{adhd} symptoms at baseline\footnote{Baseline severity is given by the pre-treatment score. 
	But, to have comparable severity scores between the different clinical scales, pre-treatment score is normalized, ie divided by the maximal score 
	which can be obtained on the clinical scale}, and the degree of engagement with \gls{nfb} intervention};
  \item \emph{the \gls{nfb} implementation}: the protocol used (\gls{scp}, \gls{smr}, 
  theta up, beta up in central or frontal areas, theta down), the presence of a transfer phase during \gls{nfb} training, the 
	possibility to train at home or at school with a transfer card, 
  the type of thresholding for discrete reward, the number of \gls{nfb} sessions, the length and frequency of the sessions, the length of
  the treatment, \textcolor{red}{the individualization of the frequency bands based on the \gls{iapf}, and coupling \gls{nfb} with \gls{emg}-Biofeedback};
  \item \emph{the acquisition quality}: the presence of one or more active electrodes and the \gls{eeg} data quality. 
  The latter was coded as an indicator between 1 and 3, using the following criteria:   
	\begin{description}
	  \item[\emph{the type of electrodes used}:] \gls{agcl}/Gel or \gls{au}/Gel;
    \item[\emph{the use of impedance mode}:] a quality check of electrode contacts
		ensuring an inter-electrode impedance smaller than $40k\Omega$;  
    \item[\emph{the level of hardware certificate}:] compliance with ISO-60601-2-26 \citep{ISO-60601-2-26:2012}.
	\end{description}
	A quality score equal to 3 was assigned if all the above criteria were satisfied. If at least one was satisfied
	the quality score was set to 2, otherwise the score was set to 1.
  \item \emph{the signal quality}: \textcolor{red}{online rejection (epoch rejected, feedback not computed) or correction (feedback computed on the denoised epoch) of \gls{eog} artifacts},
	and online rejection of generic artifacts using an amplitude-based detection.
\end{itemize}	

To prevent any bias in the analysis, the names of the factors were hidden during the entire analysis so that the data scientists (AB, QB,
DO, and LM) were fully blind to them. The names were revealed only when the data analysis and results were accepted as valid: 
this included choice of variable normalization and validation of model hypothesis, as detailed below.

The pre-processing of factors for the analysis included the following steps: factors for which there were too many 
missing observations arbitrarily set to more than 20\% of the total observations, were removed from the analysis. 
Furthermore, if a factor had more than 80\% similar observations it was also removed. 
\textcolor{red}{A study did not systematically correspond to an observation: when several clinical scales and/or raters were available in a study,
each couple clinical scale - rater was considered as an observation}.
Categorical variables were coded as dummies, i.e., the presence of the factor was represented with 1 and its absence with 0. All variables 
were standardized by subtracting the mean and then dividing by the standard deviation (not applied before the decision tree described below).

\subsubsection{Explaining within effect sizes with factors}

To compute the within-\gls{es}, the  means of total \gls{adhd} scores given by parents and teachers were used. Moreover, 
in case studies providing results for more than one behavioral scale the within-\gls{es} scores were computed for each one as:
\begin{equation*}
\label{eq:factors_effect_size_within_subject}
\text{ES} = \frac{M_{\text{post,T}} - M_{\text{pre,T}}}{\sqrt{\frac{\sigma_{\text{pre,T}}^2 + \sigma_{\text{post,T}}^2}{2}}},
\end{equation*} 
\noindent where $M_{\text{t,T}}$ is the mean of clinical scale, for treatment T, taken at time t (pre-test or post-test) and $\sigma_{\text{t,T}}$ represents 
its standard deviation. With this definition, we focus on the effect of the treatment within a group \citep{Cohen1988} as commonly reported 
in the literature \citep{Arns2009, Maurizio2014, Strehl2017}. This within-\gls{es} enables us to quantify 
the efficacy of \gls{nfb} inside the treatment group. 
\textcolor{red}{Finally, to avoid to break analysis methods assumptions, an outlier rejection was applied defining thresholds of acceptance as 
$[\mu - 3 \sigma, \mu + 3 \sigma]$, with $\mu$ and $\sigma$ being respectively the mean and the standard deviation of all within-\gls{es} computed }\citep{Shewhart1931}.

The within-\gls{es} was then considered as a dependent variable to be explained by the factors (the independent variables). 
The following three methods, implemented with the Scikit-Learn Python \citep[version 0.18.1]{Pedregosa2011} and the Statsmodels Python
\citep[version 0.8.0]{Seabold2010} libraries, were used to perform the regression:
\begin{itemize}
  \item weighted multiple linear regression with \gls{wls} \citep{Montgomery2012};
	\item sparsity-regularized linear regression with \gls{lasso} \citep{Tibshirani1996};
	\item decision tree regression \citep{Quinlan1986}.
\end{itemize}

The aim of the linear regression is to estimate the regression coefficients linking the factors
to the within-\gls{es}. A significant coefficient (here and hereafter meaning significantly different from zero) indicates
that the associated factor has an influence on \gls{nfb} efficacy and its sign the direction of the effect.
%
The \gls{wls} differs from a traditional linear regression estimated with \gls{ols} in that a weight is assigned 
to each observation in order to account for the multiplicity of reported clinical endpoints in some studies. In addition, the 
weight was also set as a function of the sample size to account for variations in sample sizes. Specifically, the weight of each study 
was taken as the ratio between the experimental group's sample size and the number of behavioral scales available.
We also ran the analysis using the \gls{ols} method to assess the impact of the weights on the results. 

The second linear method applied was the \gls{lasso}, which naturally incorporates variable selection 
into the linear model thanks to $\ell_1$-norm applied on the coefficients. A coefficient not set to zero means that 
the associated factor has an influence on \gls{nfb} efficacy and its sign indicates the direction of the effect.

The last method used to determine factors influencing \gls{nfb} was the decision tree \citep{Quinlan1986}, a hierarchical 
and non-linear method. This breaks down a dataset into smaller and smaller subsets using, at each iteration, a variable and 
a threshold chosen to optimize a simple \gls{mse} criterion \citep{James2013}. A tree is composed of several nodes and leafs, 
the importance of which decreases from the top node, called the root node, downward. 

Given that these methods are intrinsically different we compared their results. For instance, the decision
tree captures variable interactions and can relate factors to within-\gls{es} in a non-linear fashion. On the other hand, the
\gls{lasso} offers an elegant mathematical framework for variable selection. Further details are provided in the Supplemental Material
\citep{Supplementalmaterial}.


%1625













