% adding the line below for Multifile document support with LatexTools Sublime package 
%!TEX root = manuscript.tex

% Materials and Methods

\section{Materials and Methods}

\subsection{Inclusion criteria}

Search terms were directly taken from \citet{Cortese2016} to the exception of the need for a control arm, 
which is detailed in Supplemental Materials \citep{Supplementalmaterial}. The requirements included:
\begin{itemize}
	\item studies have to assess \gls{nfb} efficacy; 
	\item subjects must have received a diagnosis of \gls{adhd} based on DSM-IV \citep{DSM-4}, DSM-V \citep{DSM-5}, 
	ICD-10 \citep{ICD101993} criteria or by an expert psychiatrist; 
	\item studies have to be written in English, German, Spanish, or French;
	\item studies have to include at least eight subjects in each group;
	\item patients must be younger than 25 years old;
  \item the publication (or subsequently their corresponding author) have to disclose sufficient details on the data
	to compute required metrics for the ensuing analysis.
\end{itemize} 
The studies satisfying all these points were included in the \gls{saob}.
In order to replicate and update \citeauthor{Cortese2016}'s meta-analysis, we applied the original inclusion criteria of 
their meta-analysis to our search (the main difference being the presence of a control group). 

\subsection{Outcome definition} 

In the included studies, the severity of \gls{adhd} symptoms have been assessed by parents and, whenever available, 
by teachers. \citet{Cortese2016} and \citet{Micoulaud2014} defined parents as \gls{mprox} raters who were 
not blind to the treatment, as opposed to teachers who were considered as \gls{pblind} raters. 
This distinction is meant to assess the amplitude of the placebo effect, where it is hypothesized that teachers, 
who are presumed more blind to the intervention, are less influenced in their assessment. 
Efficacy of \gls{nfb} was measured with clinical scales, such as the \gls{adhd}-RS \citep{Pappas2006}, 
on the following outcomes: inattention, hyperactivity/impulsivity, and total scores. The factor analysis was 
performed using the total score.

\subsection{Meta-analysis}

The goal of a meta-analysis is to aggregate results from different clinical investigations and offer a 
consolidated state of evidence. To do so, it is necessary to assume 
some level of homogeneity in the design of the studies: inclusion criteria, and the presence and type of control 
(active, semi-active, or non-active). Because studies occasionally 
use slight variations of a clinical scales and because of the clinical heterogeneity of patients and control, 
the scores are standardized before being pooled into a \gls{se}. The between-\gls{es} is one of such standardized metrics, 
which we have implemented in this paper (see Supplemental Materials \citep{Supplementalmaterial}). 

The meta-analysis was performed with a Python package developed for this work. The package offers a transparent 
approach to the choice of parameters, in an effort to ease replicability. We have benchmarked it against RevMan version 5.1 
\citep[UK, London]{RevMan} by replicating \citet{Cortese2016}'s work. 
The code is made fully available on a GitHub repository \citep{Bussalb2018}, including all raw data we have used
in the present study so as to allow peers to review its implementation, update it, or use it for different projects. 
 
Before updating the \citet{Cortese2016}'s work with recently published studies
\citep{Strehl2017, Baumeister2016}, we decided to run a sensitivity analysis investigating the choices 
that later proved controversial \citep{Micoulaud2016}. Altogether, the changes investigated included:
\begin{itemize}
\item the \gls{es} of \citeauthor{Arnold2014}'s study was computed from the post-test clinical values taken 
after the completion of the 40 sessions, in contrast to \citet{Cortese2016}'s report which used the results 
after only 12 sessions because the end point values were not available at the time of his study;
\item the \gls{es} computed from the teachers' assessment reported by \citet{Steiner2014} relied on the BOSS 
Classroom Observation \citep{Shapiro2010}. This is an atypical scale to quantify \gls{adhd} symptoms since 
the Conners Rating Scale Revised \citep{Conners1998, Christiansen2014, Bluschke2016}, a well-defined
\citep{Collett2003, Epstein2012}, and broadly used metric, was available in this study. Thus, we decided 
to compute the \gls{es} based on the Conners-3, already used in this study to compute the 
parents' \gls{es}.  
\end{itemize} 

We also performed two subgroups analysis with the two choices described above: 
first, including only studies following standard protocol as defined by \citet{Arns2014} and then with studies 
whose participants took low-dose or no medication during the trial.  

\subsection{Identify factors influencing the Neurofeedback}

While revisiting the existing meta-analyses, it became apparent to us that the studies pooled together were highly heterogeneous 
in terms of methodological and practical implementation. For instance, all \gls{nfb} 
interventions were pooled together regardless the quality of the acquisition, the quality of EEG data, and the trained 
neuromarker. Likewise, the methodological implementations varied significantly, requiring the 
'subgroup' analysis (gathering studies following standard protocols for instance) that are somewhat arbitrary. To circumvent these limitations, we 
implemented a novel approach: the \gls{saob}. With this method, the within-\gls{es} of each intervention was considered 
as a dependent variable to be explained by methodological and technical factors. The results 
of such analysis should enable to identify known methodological biases (e.g.\ blind assessments negatively associated 
with \gls{es}) and possibly technical factors (e.g.\ a good control on real time data quality influences positively the 
treatment outcome). 

\subsubsection{Identify and pre-process factors}

We classified the factors influencing the efficacy of \gls{nfb} in five categories: methodological, technical,
demographics, and quality of the signal acquisition. 
Factors were chosen based on what has been reported in the literature, presumed to influence \gls{es}, 
and categorized as follow:

\begin{itemize}
  \item \emph{the signal quality}: correction of ocular and generic (amplitude based) artifacts;
  \item \emph{the population}: intake of psychostimulants during \gls{nfb} treatment and the age range of children
  included;
  \item \emph{the methodological biases}: the presence of a control group, the blindness of assessors, 
  the randomization of subjects in controlled trials, and the approval of the study by an \gls{irb};
  \item \emph{the \gls{nfb} implementation}: the protocol used (\gls{scp}, \gls{smr}, 
  theta up, beta up in central areas, theta down), the presence of a transfer phase during \gls{nfb} training, the 
	possibility to train at home or at school with a transfer card, 
  the type of thresholding for discrete reward, the number of \gls{nfb} sessions, the length and frequency of the sessions, and the length of
  the treatment;
  \item \emph{the acquisition quality}: the presence of one or more active electrodes and the \gls{eeg} data quality. 
  This latter was coded as an indicator between 1 and 3, using the following criteria:   
	\begin{description}
	  \item[\emph{the type of electrodes used}:] \gls{agcl}/Gel or \gls{au}/Gel;
    \item[\emph{the use of impedance mode}:] a quality check of electrode contacts
		ensuring an inter-electrode impedance smaller than $40k\Omega$;  
    \item[\emph{the level of hardware certificate}:] compliance with ISO-60601-2-26 \citep{ISO-60601-2-26:2012}.
	\end{description}
	A quality score equal to 3 was assigned if all the above criteria were satisfied. If at least one was satisfied
	the quality score was set to 2, otherwise the score was set to 1.
\end{itemize}	

We provide in a Github repository \citep{Bussalb2018} the raw data extracted from the publications. To prevent any
bias in the analysis, the name of the factors was hidden during the entire analysis so that the data scientists (AB, QB,
DO, and LM) were fully blind to them. The names were revealed only once the data  analysis and results were accepted as valid: 
this included choice of variable normalization and validation of model hypothesis as detailed below.

The pre-processing of factors for the analysis included the following steps: factors for which there were too many 
missing observations, arbitrarily set to more than 20\% of the total of observations, were removed from the analysis. 
Furthermore, if a factor had more than 80\% similar observations it was removed as well. Categorical variables were 
coded as dummies meaning that the presence of the factor was represented with 1 and its absence 0. All variables 
were standardized by subtracting the mean and then dividing by the standard deviation (not applied before the decision tree).

\subsubsection{Explaining effect sizes with factors}

To compute the within-\gls{es}, the  means of total \gls{adhd} scores given by parents and teachers were used. Besides, 
in case studies providing results for more than one behavioral scale, the \gls{es} scores were computed for each one as 

\begin{equation*}
\label{eq:factors_effect_size_within_subject}
\text{ES} = \frac{M_{\text{post,T}} - M_{\text{pre,T}}}{\sqrt{\frac{\sigma_{\text{pre,T}}^2 + \sigma_{\text{post,T}}^2}{2}}},
\end{equation*} 
where $M_{\text{t,T}}$ is the mean of clinical scale, for treatment T, taken at time t (pre-test or post-test) and $\sigma$ represents 
its standard deviation.

With this definition, we focused on the effect of the treatment within a group \citep{Cohen1988} as commonly reported 
in the literature \citep{Arns2009, Maurizio2014, Strehl2017}. This \gls{es} enables to quantify 
the efficacy of \gls{nfb} inside the treatment group. 

The \gls{es} was then considered as a dependent variable to be explained by the factors (the independent variables). 
The following three methods, implemented with the Scikit-Learn Python \citep[version 0.18.1]{Pedregosa2011} and the Statsmodels Python
\citep[version 0.8.0]{Seabold2010} libraries, were used to perform the regression:
\begin{itemize}
  \item weighted multiple linear regression (\gls{wls}) \citep{Montgomery2012};
	\item sparsity-regularized linear regression with \gls{lasso} \citep{Tibshirani1996};
	\item decision tree \citep{Quinlan1986}.
\end{itemize}

The aim of the linear regression is to estimate the regression coefficients linking the factors
to the \gls{es}. A significant coefficient (here and hereafter meaning significantly different from zero) indicates
that the associated factor has an influence on \gls{nfb} efficacy and its sign the direction of the effect.

The \gls{wls} differs from a traditional linear regression estimated with \gls{ols} in that a weight is assigned 
to each observation so as to account for the multiplicity of reported clinical endpoints in some studies. Besides, the 
weight was also set as a function of the sample size to account for variations in sample sizes. Specifically, the weight of each study 
was taken as the ratio between the experiment group's sample size and the number of behavioral scales available.
We also ran the analysis with \gls{ols} method to assess the impact of the weights on the results. 

The second linear method applied was the \gls{lasso}, which naturally incorporates variable selection 
to the linear model thanks to L1-norm applied on the coefficients. A coefficient not set to zero means that 
the associated factor has an influence on \gls{nfb} efficacy and its sign indicates the direction of the effect.

The last method used to determine factors influencing \gls{nfb} is the decision tree \citep{Quinlan1986}, a hierarchical 
and non-linear method. It breaks down a dataset into smaller and smaller subsets using, at each iteration, a variable and 
a threshold chosen to optimize a simple \gls{mse} criterion \citep{James2013}. A tree is composed of several nodes and leafs, 
the importance of which is decreasing from the top node, called the root node, downward. 

These methods are intrinsically different from each others, so we compared their results. For instance, the decision
tree captures variable interactions and can relate factors to \gls{es} in a non-linear fashion. On the other hand, the
\gls{lasso} offers an elegant mathematical framework to variable selection. Further details are given in the Supplemental Material
\citep{Supplementalmaterial}.


%1636













