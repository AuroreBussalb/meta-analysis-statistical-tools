\section*{Table captions}

\begin{table}[h!]
  \centering
  \caption{List of all studies included in the three different analysis.}
  \fontsize{9}{11}\selectfont
\begin{tabular}{ cccccc }
\toprule
\multicolumn{3}{ c }{Dataset} & Study & Year & \shortstack{ Size of the \\ \gls{nfb} group } \\
\midrule
 & & & \citeauthor{Arnold2014} & 2014 & 26 \\ 
 & & & \citeauthor{Bakhshayesh2011} & 2011 & 18 \\
 & & & \citeauthor{Beauregard2006} & 2006 & 15 \\
 & & & \citeauthor{Bink2014} & 2014 & 45 \\
 & & & \citeauthor{Christiansen2014} & 2014 & 14 \\
 & & & \citeauthor{Gevensleben2009} & 2009 & 59 \\
 & & & \citeauthor{Heinrich2004} & 2004 & 13 \\
 & & & \citeauthor{Holtmann2009} & 2009 & 20 \\
 & & & \citeauthor{Linden1996} & 1996 & 9 \\
 & & & \citeauthor{Maurizio2014} & 2014 & 13 \\
 & & & \citeauthor{Steiner2011} & 2011 & 9 \\
 & & & \citeauthor{Steiner2014} & 2014 & 34 \\
 & & |\shortstack{ Replicate \\ \citeauthor{Cortese2016}$^a$ } & \citeauthor{VanDongen2013} & 2013 & 22 \\
\cmidrule(lr){3-6}
 & & & \citeauthor{Baumeister2016} & 2016 & 8 \\
 & \shortstack{ Update \\ \citeauthor{Cortese2016}$^b$ } & & \citeauthor{Strehl2017} & 2017 & 72 \\
\cmidrule(lr){2-6}
 & & & \citeauthor{Bluschke2016} & 2016 & 19 \\
 & & & \citeauthor{Deilami2016} & 2016 & 12 \\
 & & & \citeauthor{Drechsler2007} & 2007 & 17 \\
 & & & \citeauthor{Duric2012} & 2012 & 23 \\
 & & &\citeauthor{Escolano2014} & 2014 & 20 \\
 & & & \citeauthor{Fuchs2003} & 2003 & 22 \\
 & & & \citeauthor{Kropotov2005} & 2005 & 86 \\
 & & & \citeauthor{Lee2017} & 2017 & 18 \\
 & & & \citeauthor{Leins2007} & 2007 & 19 \\
 & & & \citeauthor{Li2013} & 2013 & 20 \\
 & & & \citeauthor{Meisel2014} & 2014 & 12 \\
 & & & \citeauthor{Mohagheghi2017} & 2017 & 30 \\
 & & & \citeauthor{Mohammadi2015} & 2015 & 16 \\
 & & & \citeauthor{Monastra2002} & 2002 & 51 \\
 & & & \citeauthor{Ogrim2013} & 2013 & 13 \\
 \gls{saob}$^c$ & & & \citeauthor{Strehl2006} & 2006 & 23 \\
\bottomrule
\end{tabular}

	\begin{tablenotes}
	\item $^a$ Studies originally included in \citet{Cortese2016}
	(search on August 30, 2015), $^b$ studies satisfying \citet{Cortese2016}'s criteria (search on February 12, 2018), $^c$ studies 
	satisfying \citet{Cortese2016}'s criteria to the exception of the part relative to the control group (search on February 12, 2018).
	\end{tablenotes}
  \label{Table:table_factors_analysis_meta_analysis_list_studies}
\end{table}

\begin{table}[h!]
  \centering
  \caption{Comparison between \citet{Cortese2016} results obtained with RevMan \citep{RevMan} and those obtained with the Python code with our 
	choices applied. $\glspl{se}$ and their corresponding p-value (in parenthesis) are presented. With the Python program, a negative $\gls{se}$
	is in favor of \gls{nfb} unlike \citeauthor{Cortese2016}.}
  \begin{tabular}{cccc}

\toprule
\multicolumn{2}{c}{Input data} & \shortstack{ Results from \\ \citet{Cortese2016} \\ (for reference)} & \shortstack{ Means and standard \\ deviations from \\ articles included in \\ \citet{Cortese2016} }\\
\midrule
\multicolumn{2}{c}{Implementation} & \shortstack{ RevMan \\ \citet{RevMan} } & Python program\\
\midrule
\multicolumn{2}{ c }{Hypothesis} & \shortstack{ Same as \\ \citet{Cortese2016} } & Our choices$^a$\\
\midrule
\multirow{ 3}{*}{ \textit{Parents} } & Total & $0.35$ ($0.004$) & $-0.32$ ($0.013$)\\
 & Inattention  & $0.36$ ($0.009$) & $-0.31$ ($0.036$)\\
 & Hyperactivity  & $0.26$ ($0.004$) & $-0.24$ ($0.02$)\\
\multirow{ 3}{*}{ \textit{Teachers} } & Total & $0.15$ ($0.20$) & $-0.11$ ($0.37$)\\
 & Inattention  & $0.06$ ($0.70$) & $-0.17$ ($0.16$)\\
 & Hyperactivity  & $0.17$ ($0.13$) & $-0.022$ ($0.85$)\\
\bottomrule

\end{tabular}

	\begin{tablenotes}
	\item $^a$ post-test values for \citeauthor{Arnold2014} are obtained after 40 sessions of \gls{nfb} and Conners scale is used for \citeauthor{Steiner2014}
	teachers' outcomes.
	\end{tablenotes}
  \label{Table:meta_review_comparison_revman_and_python_with_choices}
\end{table}

\begin{table}[h!]
  \centering
  \caption{Results of the \gls{wls}, \gls{lasso} and decision tree. For the \gls{wls}, a p-value $<$ 0.05 (in bold) means that the coefficient of 
	the corresponding factor is significantly different from 0. For the \gls{lasso}, factors not set to 0 (in bold) are selected. For the decision tree,
	the place of the factor in the tree is precised. When the value of the coefficient is negative, the corresponding factor may lead to better \gls{nfb} results.}
  \begin{center}
\begin{tabular}{ p{3cm} p{3cm} p{3cm} p{2cm} p{2cm} p{2cm}}
\toprule
\multicolumn{2}{c}{ \shortstack{Independent \\ variables (factors)} } & \shortstack{ \gls{wls} \\ (p-value) } & \gls{lasso} & \shortstack{Decision \\ Tree} \\
\midrule
\multirow{ 2}{*}{ \textit{Signal quality} } & \gls{eog} correction & \hskip 0.08in-0.078 (0.41) &  \hskip 0.12in0.00 & / \\ 
& artifact correction based on amplitude & \hskip 0.12in\textbf{0.15(0.040)} & \hskip 0.12in\textbf{0.047} & / \\ 
\midrule
\multirow{ 3}{*}{ \textit{Methodological} } & \gls{pblind} & \hskip 0.12in\textbf{0.10 (0.043)} & \hskip 0.12in\textbf{0.11} & \textbf{selected} \\ 
& randomization & \hskip 0.12in0.0069 (0.92) & \hskip 0.12in\textbf{0.032} & / \\  
& \gls{irb} & \hskip 0.08in\textbf{-0.29 (0.00)} & \hskip 0.08in\textbf{-0.15} & / \\  
\midrule
\multirow{ 3}{*}{ \textit{Population} } & age max & \hskip 0.08in-0.090 (0.16) & \hskip 0.12in0.00 & / \\
& age min & \hskip 0.08in-0.055 (0.37) & \hskip 0.12in0.00 & \textbf{selected }\\
& on drugs & \hskip 0.12in0.069 (0.42) & \hskip 0.12in\textbf{0.032} & / \\
\midrule
\multirow{ 9}{*}{ \textit{ \shortstack{\gls{nfb} \\ implementation} } } & number of sessions  & \hskip 0.08in-0.0075 (0.92) & \hskip 0.12in0.00 & / \\
& session length & \hskip 0.12in0.17 (0.17) & \hskip 0.12in0.00 & \textbf{selected} \\
& treatment length & \hskip 0.12in\textbf{0.57 (0.00)} & \hskip 0.12in\textbf{0.33} & \textbf{selected} \\
& session pace & \hskip 0.08in\textbf{-0.25 (0.00)} & \hskip 0.08in\textbf{-0.14} & / \\ 
& \gls{smr} & \hskip 0.08in-0.063 (0.41) & \hskip 0.12in\textbf{0.061} & \textbf{selected} \\
& beta up central & \hskip 0.08in-0.027 (0.72) & \hskip 0.12in0.00 & / \\  
& theta down & \hskip 0.08in\textbf{-0.29 (0.014)} & \hskip 0.08in\textbf{-0.051} & / \\
& \gls{scp} & \hskip 0.08in-0.099 (0.50) & \hskip 0.12in\textbf{0.10} & / \\ 
& transfer phase & \hskip 0.12in\textbf{0.27 (0.032)} & \hskip 0.12in\textbf{0.11} & / \\
\midrule
\multirow{ 2}{*}{ \textit{ \shortstack{Quality of \\ acquisition} } } & more than one active electrode & \hskip 0.12in0.064 (0.36) & \hskip 0.12in0.00 & / \\ 
& \gls{eeg} quality 2 & \hskip 0.08in\textbf{-0.36 (0.00)} & \hskip 0.08in\textbf{-0.23} & \textbf{selected} \\  
\bottomrule
\end{tabular}
\end{center}

  \label{Table:table_factors_analysis_results_summary}
\end{table}