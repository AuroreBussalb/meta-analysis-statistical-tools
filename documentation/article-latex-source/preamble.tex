% adding the line below for Multifile document support with LatexTools Sublime package 
%!TEX root = manuscript.tex


\usepackage[utf8]{inputenc}
\usepackage[english]{babel}

% Quotes 
\usepackage[square]{natbib}
\bibliographystyle{abbrvnat}

% For hyperlinks in the pdf 
\usepackage{hyperref}

% Glossaries
%\usepackage[acronym]{glossaries}
\usepackage[acronym,shortcuts]{glossaries}

% Font Helvetica
\renewcommand{\familydefault}{\sfdefault}
\usepackage[T1]{fontenc}

% Margins
\usepackage{geometry}
 \geometry{
 a4paper,
 total={170mm,257mm},
 left=20mm,
 top=20mm,
 }

% Linespace
\linespread{1.5}

% For pictures in the pdf
\usepackage{graphicx}

% For tables
\usepackage{multirow}
\usepackage{booktabs}
\usepackage{threeparttable}

% For ref with figure, table or equation before the number
\usepackage{cleveref}

% landscape page
\usepackage{lscape}

% for argmin and argmax
\usepackage{amsmath}
\DeclareMathOperator*{\argmin}{argmin}

% Glossary 
\usepackage[acronym]{glossaries}
%\makeglossaries

% Captions
\usepackage{booktabs,caption}

% Color
\usepackage{xcolor}

% Cross out words
\usepackage[normalem]{ulem}

% This is for me to comment

% Not using the pdfcomment package but it is an interesting one  
%\usepackage[author={Louis Mayaud}]{pdfcomment}

% Select what to do with todonotes: 
% \usepackage[disable]{todonotes} % notes not showed
\usepackage[draft]{todonotes}   % notes showed

% Select what to do with command \comment:  
% \newcommand{\comment}[1]{}  %comment not showed
\newcommand{\comment}[1]
{\par {\bfseries \color{blue} #1 \par}} %comment showed
