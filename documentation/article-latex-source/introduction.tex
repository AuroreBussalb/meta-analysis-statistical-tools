% adding the line below for Multifile document support with LatexTools Sublime package 
%!TEX root = manuscript.tex

% Introduction

\section{Introduction} 

\gls{adhd} is a common child psychiatric disorder characterized by impaired attention and/or hyperactivity/impulsivity.
Symptoms may persist in adulthood with clinical significance, which makes \gls{adhd} a life-long problem for many
patients \citep{Faraone2006}. The prevalence of \gls{adhd} is about 5\% in school-aged children yielding to an estimated
2.5 millions of children in Europe \citep{DSM-5}. \gls{adhd} impacts children's well-being with many of them suffering
from low self esteem \citep{Shaw2005} and underachieve at school \citep{Barry2002}. Parents are equally affected by this
situation \citep{Harpin2005}. Besides \gls{adhd} has a financial impact: a survey of 2013 in the Netherlands estimated
at between $9,860$ and $14,483$\euro the costs related to \gls{adhd} per patient annually \citep{le2014}. 
 
The diagnosis of \gls{adhd} primarily relies on questionnaire-based clinical evaluation \citep{DSM-5}, which can be
supported with objective assessment metrics of executive function such as the \gls{tova} \citep{Forbes1998}, the
\gls{cpt} \citep{Barkley1991}, and the \gls{sart} \citep{Robertson1997}. Conversely, objective markers of brain function
using \gls{eeg}, \gls{fmri}, or \gls{pet} could not successfully improve diagnosis \citep{Neba} at the individual
level but proved significantly different at the population level.  For instance, \gls{adhd} patients were found to show
an increase in theta waves (4-8Hz) in the frontal area whereas there are less beta waves (12-32Hz) and \gls{smr}
(13-15Hz) in the central area \citep{Monastra2005, Matouvsek1984, Janzen1995,loo2017}.  
 
Among all existing treatments, the most widely used is psychostimulants, which have been proven to be efficacious
\citep{Taylor2014, Storebo2015}. However, its long-term effectiveness is still an active area of research
\citep{DuPaul1998, Swanson2001, Jensen1999}. Moreover, \gls{adhd} children under medication commonly suffer from mild
side effects such as loss of appetite and sleep problems even though only few serious adverse events have been reported
\citep{Storebo2015, Cooper2011}. These drawbacks make some parents and clinicians reluctant to choose such treatment
turning them to drug-free alternatives such as dietary changes \citep{Belanger2009} and behavioral therapy, which have
been proven less efficacious \citep{Sonuga-Barke2013}.

\gls{nfb} is a noninvasive technique aiming at the reduction of \gls{adhd} symptoms \citep{Arns2015, Steffert2010,
Marzbani2016}. It is a self-paced brain neuromodulation technique that represents one's brain activity in real-time
using auditory or visual modulations, on which learning paradigms can be applied such as operant conditioning
\citep{Reynolds1975} or voluntary control. To deliver this intervention, neurophysiological time series must be recorded
and analyzed in real-time and implemented in serious games leveraging learning paradigms \citep{Wang2010}. To that
effect, recorded brain signals are analyzed to extract a real time representation of the activity of a population of
neurons involved in attentional networks, which is translated into visual or auditory cues. The sensory feedback
constitutes the rewards mechanism promoting learning using operant conditioning protocol \citep{Sherlin2011}. Operant
conditioning enables neural plasticity supporting the child in the task repetition \citep{Skinner1961} leading to
lasting neuronal reorganization \citep{VanDoren2017}. 

In case of \gls{adhd}, several \gls{nfb} protocols have been proposed and investigated to decrease the symptoms:
\begin{itemize} 
  \item protocols based on frequency-band training: a child can be asked to enhance his \gls{smr} while
    suppressing theta or beta \citep{Lubar1976}, or enhance beta while suppressing theta (this scenario is known as
    \gls{tbr}) \citep{Arns2013}; 
  \item protocol based on the \glspl{scp} training consisting in the regulation of
    cortical excitation thresholds by focusing on activity generated by external cues (similar to \glspl{erp})
    \citep{Heinrich2004, Banaschewski2007}; 
  \item protocol based on \glspl{erp} (P300) \citep{Fouillen2017}: children
    have to focus on external cues leading to a reduced P300 amplitude so it can be considered as a specific
    neurophysiological marker of selective attention \citep{RolandLeBouedec}.  
\end{itemize} 

Shortly after the discovery of the brain's electric activity by \citet{Berger1929}, \citet{Durup1935} proved it could be
voluntarily modulated leading a series of finding on the self-regulation of brain activity. The first indication of its
therapeutic potential came forty years later when \citet{Sterman1974} serendipitously found the training of \gls{smr}
activity to reduce the incidence of epileptic crisis in kerozen-exposed cats. The technique, then known as \gls{nfb}
quickly became investigated in various fields of neuropsychiatry including, most notably, \gls{adhd} and resulting in a
relatively large body of scientific literature \citep{Lubar1976, Rossiter1995, Linden1996, Maurizio2014}. Subsequently,
its efficacy on the core symptoms of \gls{adhd} (inattention, hyperactivity, and impulsivity) has been subject to several
meta-analytic studies \citep{Loo2005, Lofthouse2012, Arns2009, Micoulaud2014, Sonuga-Barke2013}. 

The most recent meta-analysis solely on the efficacy of \gls{nfb} has been conducted by \citet{Cortese2016} where 13
studies were included. Although only \glspl{rct} were selected, the authors of this meta-analysis made some choices that
have since been debated by the community. Specifically, \citet{Micoulaud2016} criticized the use of an uncommon
behavioral scale provided by \citet{Steiner2014} for the teachers' assessments and the inclusion of a pilot study
carried out by \citet{Arnold2014}. 

Finally, because of the publication of new \glspl{rct} meeting \citeauthor{Cortese2016}'s inclusion criteria, it was
decided to update his work and take the opportunity to investigate some choices that later proved controversial.
Eventually, we extended the analysis with a novel method: the \gls{saob} that takes advantage of studies technical and
methodological high heterogeneity rather than suffering from it. Indeed, the \gls{nfb} domain is characterized by a
clinical literature that is tremendously heterogeneous: studies differ methodologically (randomization and presence of a
blind assessor for instance) but also on the \gls{nfb} implementation (number of sessions, session and treatment length,
and type of protocol for instance) as well as on the acquisition and pre-processing of the \gls{eeg}. Since
methodological and technical implementations of studies may influence their outcomes, we suggest here to identify which
of the factors independently influence the reported \gls{es} with the use of adequate statistical tools.

% number of words: 826





