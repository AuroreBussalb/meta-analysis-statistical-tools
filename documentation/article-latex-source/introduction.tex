% adding the line below for Multifile document support with LatexTools Sublime package 
%!TEX root = manuscript.tex

% Introduction

\section{Introduction} 

% What is ADHD? What are its consequences?
\glsfirst{adhd} is a common childhood psychiatric disorder characterized by impaired attention and/or hyperactivity/impulsivity.
Symptoms may persist in adulthood with clinical significance, which makes \gls{adhd} a life-long problem for many
patients \citep{Faraone2006}. The prevalence of \gls{adhd} is around 5\% in school-aged children, thus affecting an estimated
2.5 million children in Europe \citep{DSM-5}. \gls{adhd} negatively impacts children's well-being, with many suffering
from low self-esteem \citep{Shaw2005} and underachievement in school \citep{Barry2002}. Parents are equally affected, since 
the child's behavior is frequently attributed to bad parenting \citep{Harpin2005}. From a societal point
of view, \gls{adhd} also has a high financial impact: a 2013 survey in Europe estimated costs related to \gls{adhd} between $9,860$ and 
$14,483$ Euros per patient/year \citep{le2014}. 

% How is it diagnosed?
The diagnosis of \gls{adhd} primarily relies on questionnaire-based clinical evaluation \citep{DSM-5}, which can be
supported by objective assessment metrics of executive function such as the \gls{tova} \citep{Forbes1998}, the
\gls{cpt} \citep{Barkley1991}, and the \gls{sart} \citep{Robertson1997}. Objective markers of brain function
using \gls{eeg}, \gls{fmri}, or \gls{pet} are not considered to be useful for improving diagnosis at the individual
level, but can help in differentiating groups of patients \citep{Johnstone2005}.  
In particular, different phenotypes of \gls{adhd} patients present with an increase in the \gls{eeg} theta wave 
power (4-8Hz) and/or a decrease of \gls{eeg} beta wave power (12-32Hz) in frontal areas, or a decrease in the \gls{eeg} 
\gls{smr} power (13-15Hz) in the central area \citep{Monastra2005, Matouvsek1984, Janzen1995, loo2017}. A device 
using \gls{eeg} to help clinicians more accurately diagnosis \gls{adhd} was cleared by the \gls{fda} \citep{Neba}. 

% How is it treated? 
Psychostimulants are the most common treatment currently in use, and have proven to be efficacious
\citep{Taylor2014, Storebo2015}. However, their long-term effectiveness and side effects are still debated and form 
an active area of research \citep{DuPaul1998, Swanson2001, Jensen1999, Su2016, Becker2016}. Moreover, \gls{adhd} children under medication 
commonly suffer from mild side effects such as loss of appetite and sleep disturbance, although serious adverse events
are rare \citep{Storebo2015, Cooper2011}. These drawbacks make some parents and clinicians reluctant to 
opt for such treatment, instead turning to non-pharmaceutical alternatives such as dietary changes \citep{Belanger2009} and behavioral 
therapy, which have been proven to be less efficacious \citep{Sonuga-Barke2013}.

% What is NFB? 
\Glsfirst{nfb} is another non-pharmaceutical and non-invasive approach aiming at the reduction of \gls{adhd} symptoms 
\citep{Arns2015, Steffert2010, Marzbani2016}. Shortly after the discovery of the brain's electric activity by 
\citet{Berger1929}, \citet{Durup1935} demonstrated it could be voluntarily modulated, leading to a series of findings on the 
self-regulation of brain activity. The first indication of the therapeutic potential of brain activity operant-conditioning 
came forty years later when \citet{Sterman1974} found that training the \gls{smr} activity reduces the incidence 
of epileptic crisis in kerosene-exposed cats. The technique, then known as \gls{nfb}, rapidly became the subject of investigation in various 
fields of neuropsychiatry including, most notably, \gls{adhd} \citep{Lubar1976, Rossiter1995, Linden1996, Maurizio2014}.

\Gls{nfb} is a self-paced brain neuromodulation technique that represents brain activity in real-time using auditory 
or visual modulations, on which learning paradigms, such as operant conditioning
\citep{Reynolds1975} or voluntary control, can be applied. To deliver this intervention, neurophysiological time series 
are analyzed online in order to drive feedback applications such as serious games \citep{Wang2010}. 
The signal of interest should represent the activity of a population of neurons involved in attentional networks, which is translated into 
visual or auditory cues. The sensory feedback constitutes the rewards mechanism, promoting learning using, for instance, operant conditioning 
protocols \citep{Sherlin2011}. Operant conditioning enables neural plasticity, thus supporting the child in the task repetition \citep{Skinner1961}, 
which is believed to result in long-lasting neuronal reorganization \citep{VanDoren2017}. 

Several \gls{nfb} protocols have been proposed and investigated for decreasing the symptoms of \gls{adhd}:
\begin{itemize} 
  \item protocols based on neural oscillations, using frequency-band power training: enhancing \gls{smr} \citep{Beauregard2006}, reducing theta  
	  \citep{Marzbani2016} or enhancing beta \citep{Kropotov2005}, or a composite protocol such as enhancing beta while suppressing theta, also known as the \gls{tbr}
    protocol \citep{Lubar1976, Arns2013}; 
  \item protocols based on \glspl{scp} training consisting of the regulation of
    cortical excitation thresholds by focusing on activity generated by external cues 
    \citep{Heinrich2004, Banaschewski2007}; 
  \item protocols to enhance \glspl{erp}: in particular, the amplitude of the P300 \gls{erp} can be considered as a specific
    neurophysiological marker of selective attention \citep{Fouillen2017}.  
\end{itemize} 

Moreover, \gls{nfb} protocols can be personalized: some studies did not use the usual definitions of \gls{eeg} band ranges 
but determined them thanks to the \gls{iapf} \citep{Klimesch1999}, giving individualized \gls{nfb} protocols \citep{Liu2016, Escolano2014, Bazanova2018}.

% Clinical evidence of NFB
\Gls{nfb} efficacy on the core symptoms of \gls{adhd} (inattention, hyperactivity, and impulsivity) has been the 
subject of several meta-analytic studies \citep{Loo2005, Lofthouse2012, Arns2009, Micoulaud2014, Sonuga-Barke2013}. 
To date, studies have not reached a consensus on the efficacy of \gls{nfb}; while \citet{Arns2009} and \citet{Micoulaud2014} 
claim results in favor of its efficacy, especially on the inattention component highlighted by \citeauthor{Micoulaud2014}, other authors, such as
\citet{Loo2005, Lofthouse2012}, and \citet{Sonuga-Barke2013} express their reservations, asking for further evidence from blind assessment.

% Position of the problem
The most recent meta-analysis addressing the efficacy of \gls{nfb} was published by \citet{Cortese2016}, including a total of 13
\glspl{rct}. The results of this analysis are mixed: when based on parent assessments, which are not blind to treatment, they are significantly 
in favor of \gls{nfb}, whereas when the evolution of symptoms is rated by teachers (considered as probably blind), the results are no longer 
significant. The authors concluded that further evidence from blind assessments is needed in order to support \gls{nfb} as a treatment for \gls{adhd} symptoms.
However, some of the choices made in this meta-analysis, which may have had an impact on the results, have since been debated by the community. Specifically, 
\citet{Micoulaud2016} criticized the use of an uncommon behavioral scale provided by \citet{Steiner2014} for the teachers' assessments 
and the inclusion of a pilot study carried out by \citet{Arnold2014}. 

% Work done here
As a result of these criticisms and the concurrent publication of new \glspl{rct} meeting \citeauthor{Cortese2016}'s inclusion criteria, we
decided to update this meta-analysis and take the opportunity to investigate the impact of its controversial choices.
While performing our investigation, we observed two shortcomings: the assumption that the difference between teacher and parent
assessments can solely be explained by the placebo effect, and pooling together heterogeneous studies in terms of methodology and technical
implementation. An interesting approach, albeit not commonly performed, to assess the \gls{nfb} efficacy would be to analyze the specificity of the \gls{eeg} changes with respect 
to trained neuromarkers \citep{Maurizio2014}. In our case, based on the data at our disposal, we used the technical and methodological heterogeneity 
of the \gls{nfb} trials to our advantage rather than disadvantage by extending the previous work with a novel method, the \glsfirst{saob}. Indeed, the \gls{nfb} domain is 
characterized by clinical literature that is extremely heterogeneous: studies differ methodologically (for instance, random assignment and 
presence of a blind assessment), in the \gls{nfb} implementation (for instance, number of sessions, session and treatment length,
and type of protocol) as well as on the acquisition and processing of the \gls{eeg} signal. Description and analysis of different types 
of \gls{nfb} implementation was subject to several studies \citep{Arns2014, Enriquez2017, Vernon2004, Jeunet2018}. However to the best of our knowledge, none 
of these studies has used statistical tools to quantify their influence on clinical endpoints.

Since methodological and technical implementations of studies are highly likely to influence their outcomes \citep{Congedo2004}, we suggest 
identifying which of the factors independently influence the clinical efficacy with the use of appropriate statistical tools. 
In addition, we have made available all the raw \gls{rct} data we have used and a complete Python library for performing meta-analysis \citep{Bussalb2019}. 
Through doing so, we hope to foster the replication of our and previous studies and to facilitate similar future projects.


% number of words: 1112





