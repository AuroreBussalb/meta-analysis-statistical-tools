% adding the line below for Multifile document support with LatexTools Sublime package 
%!TEX root = manuscript.tex

% Introduction

\section{Introduction} 

\gls{adhd} is a common child psychiatric disorder characterized by impaired attention and/or hyperactivity/impulsivity.
Symptoms may persist in adulthood with clinical significance, which makes \gls{adhd} a life-long problem for many
patients \citep{Faraone2006}. The prevalence of \gls{adhd} is about 5\% in school-aged children yielding to an estimated
2.5 millions of children in Europe \citep{DSM-5}. \gls{adhd} impacts children's well-being with many of them suffering
from low self esteem \citep{Shaw2005} and underachieving at school \citep{Barry2002}. Parents are equally affected by this
situation since children's behavior is commonly attributed to bad parenting \citep{Harpin2005}. From a societal point
of view, \gls{adhd} has a high financial impact: a survey of 2013 in Europe estimated at between $9,860$ and 
$14,483$ Euros per patient/year the cost related to \gls{adhd} \citep{le2014}. 
 
The diagnosis of \gls{adhd} primarily relies on questionnaire-based clinical evaluation \citep{DSM-5}, which can be
supported with objective assessment metrics of executive function such as the \gls{tova} \citep{Forbes1998}, the
\gls{cpt} \citep{Barkley1991}, and the \gls{sart} \citep{Robertson1997}. Objective markers of brain function
using \gls{eeg}, \gls{fmri}, or \gls{pet} are not considered useful to improve diagnosis at the individual
level \citep{Neba}, but can help differentiating groups of patients \citep{Johnstone2005}.  
In particular, different phenotypes of \gls{adhd} patients present with an increase in the \gls{eeg} theta waves 
power (4-8Hz) and/or a decrease of \gls{eeg} beta waves power (12-32Hz) in frontal areas, or a decrease in the \gls{eeg} 
\gls{smr} power (13-15Hz) in the central area \citep{Monastra2005, Matouvsek1984, Janzen1995, loo2017}.  
 
Among all existing treatments, the most widely used is psychostimulants, which have been proven to be efficacious
\citep{Taylor2014, Storebo2015}. However, their long-term effectiveness and side effects are still debated and form 
an active area of research \citep{DuPaul1998, Swanson2001, Jensen1999}. Moreover, \gls{adhd} children under medication 
commonly suffer from mild side effects such as loss of appetite and sleep disturbance, however serious adverse events
are rare \citep{Storebo2015, Cooper2011}. These drawbacks make some parents and clinicians reluctant to 
opt for such treatment, turning them to non-pharmaceutical alternatives such as dietary changes \citep{Belanger2009} and behavioral 
therapy, which have been proven to be less efficacious \citep{Sonuga-Barke2013}.

\gls{nfb} is a noninvasive technique aiming at the reduction of \gls{adhd} symptoms \citep{Arns2015, Steffert2010,
Marzbani2016}. It is a self-paced brain neuromodulation technique that represents one's brain activity in real-time
using auditory or visual modulations, on which learning paradigms, such as operant conditioning
\citep{Reynolds1975} or voluntary control, can be applied. To deliver this intervention, neurophysiological time series 
analyzed in real-time so as to be incorporated in feedback applications such as serious games leveraging learning paradigms \citep{Wang2010}. 
These data represent the activity of a population of neurons involved in attentional networks, which is translated into 
visual or auditory cues. The sensory feedback constitutes the rewards mechanism promoting learning using operant conditioning 
protocol \citep{Sherlin2011}. Operant conditioning enables neural plasticity supporting the child in the task repetition \citep{Skinner1961}, 
which is supposed to result in long-lasting neuronal reorganization \citep{VanDoren2017}. 

Several \gls{nfb} protocols have been proposed and investigated to decrease the symptoms of \gls{adhd}:
\begin{itemize} 
  \item protocols based on neural oscillations, using frequency-band power training: enhance \gls{smr} while
    suppressing theta power, or enhance beta while suppressing theta (this is known as
    the \gls{tbr} protocol) \citep{Lubar1976, Arns2013}; 
  \item the protocol based on the \glspl{scp} training consisting in the regulation of
    cortical excitation thresholds by focusing on activity generated by external cues (similar to \glspl{erp})
    \citep{Heinrich2004, Banaschewski2007}; 
  \item the protocol to enhance \glspl{erp} (P300) \citep{Fouillen2017}: P300 amplitude can be considered as a specific
    neurophysiological marker of selective attention \citep{RolandLeBouedec}.  
\end{itemize} 

Shortly after the discovery of the brain's electric activity by \citet{Berger1929}, \citet{Durup1935} proved it could be
voluntarily modulated leading to a series of finding on the self-regulation of brain activity. The first indication of the
therapeutic potential of brain activity operant-conditioning came forty years later when \citet{Sterman1974} serendipitously 
found that training the \gls{smr} activity reduces the incidence of epileptic crisis in kerozen-exposed cats. The technique, 
then known as \gls{nfb}, quickly became investigated in various fields of neuropsychiatry including, most notably, \gls{adhd} 
\citep{Lubar1976, Rossiter1995, Linden1996, Maurizio2014}. Subsequently, its efficacy on the core symptoms of \gls{adhd} 
(inattention, hyperactivity, and impulsivity) has been the subject of several meta-analytic studies \citep{Loo2005, Lofthouse2012, 
Arns2009, Micoulaud2014, Sonuga-Barke2013}. 

The last meta-analysis addressing the efficacy of \gls{nfb} has been published by \citet{Cortese2016}. A total of 13
studies were included. The authors of this meta-analysis have made some choices that
have since been debated by the community. Specifically, \citet{Micoulaud2016} criticized the use of an uncommon
behavioral scale provided by \citet{Steiner2014} for the teachers' assessments and the inclusion of a pilot study
carried out by \citet{Arnold2014}. 

Finally, because of the publication of new \glspl{rct} meeting \citeauthor{Cortese2016}'s inclusion criteria, we
decided to update the meta-analysis and take the opportunity to investigate the impact of the controversial choices.
We have extended the analysis with a novel method, the \gls{saob}, which takes advantage of the technical and
methodological heterogeneity of the trials included in the meta-analysis rather than suffering from it. Indeed, the \gls{nfb} domain is 
characterized by a clinical literature that is tremendously heterogeneous: studies differ methodologically (random assignment and 
presence of a blind assessment for instance), in the \gls{nfb} implementation (number of sessions, session and treatment length,
and type of protocol for instance) as well as on the acquisition and processing of the \gls{eeg} signal. Since
methodological and technical implementations of studies are very likely to influence their outcomes \citep{Congedo2004}, we suggest here 
to identify which of the factors independently influence the clinical efficacy with the use of appropriate statistical tools.

% number of words: 835





