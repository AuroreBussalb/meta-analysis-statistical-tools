% adding the line below for Multifile document support with LatexTools Sublime package 
%!TEX root = manuscript.tex


% Discussion

\section{Discussion}

\subsection{Perform the meta-analysis} 
In the meta-analysis performed here, we challenged some choices made by
\citeauthor{Cortese2016}, which proved controversial: the computation of \gls{es} based on an unusual scale
\citep{Steiner2014} and the inclusion of a pilot study \citep{Arnold2014} whose end point values were not available at
the time \citeauthor{Cortese2016} conducted his meta-analysis. We review here the list of changes, their justification,
and their impact on the analysis.
 
First, relying on the Conners-3 \citep{Conners2008} instead of the BOSS Classroom Observation \citep{Shapiro2010} for
teachers ratings seemed preferable because this scale is more commonly used \citep{Christiansen2014, Bluschke2016} and is
the revision of the Conners Rating Scale Revised \citep{Conners1998} whose reliability has been studied
\citep{Collett2003}. However, relying on one or the other scale did not change the significance of the \gls{es} whatever
the outcome studied.

Second, to compute the \gls{es} of \citet{Arnold2014} we used the clinical scores taken when all sessions
were completed instead of looking at interim results as in \citeauthor{Cortese2016}. Some studies suggested that the
number of sessions correlates positively with the changes observed in the
\gls{eeg} \citep{Vernon2004} so that a lower number of sessions would lead to artificially smaller \gls{es}. Here, the
\gls{es} computed with the values at post test of \citet{Arnold2014} were smaller than those obtained after 12 sessions
but these differences did not lead to a change of significance of the \gls{se}. 

To conclude on that meta-analysis, though some points from the original were controversial and the fact that - for the
reasons mentioned earlier - different choices could reasonably be made, it turned out that the impact on the
meta-analysis results were minimal and did not change the statistical significance of any outcome.  Consequently, the
completion of the meta-analysis with studies published since the publication of his work were done with the choices:
\begin{itemize} 
  \item to compute the \gls{es} of \citet{Arnold2014} the end point values at were used; 
  \item the scores reported by teachers on the Conners-3 in \citeauthor{Steiner2014}'s study were taken into account instead of these of the BOSS Classroom Observation.  
\end{itemize} 

The addition of the two new studies \citep{Strehl2017, Baumeister2016} further confirmed original results. Indeed, the
significance did not change for any outcome: \gls{se} found remained significant for \gls{mprox} raters and
non-significant for \gls{pblind}.  Adding two more studies increased the significance of the sensitivity analysis ran by
\citeauthor{Cortese2016}, most notably, the \gls{se} of studies corresponding to \gls{nfb} standard protocols \citet{Arns2014}. While \citeauthor{Cortese2016} found that this subset tended to perform better, particularly on the \gls{pblind} outcome, adding two studies confirmed this result on the total clinical score (p-value < 0.05).  Despite the obvious heterogeneity of the studies included in this subset (particularly in terms of protocol used), this result
suggests a positive relation between the features of this \emph{standard} design and \gls{nfb} performance.

Concerning the raters, we considered teachers as \gls{pblind} raters as
\citeauthor{Cortese2016,Micoulaud2014} did. The stress put on \emph{probably} indicated that teacher may nonetheless be
aware of the treatment followed. An element that corroborates this hypothesis is the fact that on all the studies
included in this work, the amplitude of the clinical scale at baseline suggests that teachers did not capture the full
extend of the symptoms or, say it differently, that they were more blind to the symptoms than to the intervention. 
As a consequence, teachers might just very well be less likely to observe a clinical change over the course of the
treatment (prone to type II error) \citep{Sollie2013, Narad2015,
  Minder2018}. So using \gls{pblind} as an estimate for correcting the placebo effect might not be optimal.
\todo{I understand why you split the pblind discussion over tge two methods, but I would instead suggest that you
  regroup this paragraph with the ones discussing this topic under a new subsection}

% 573 words

\subsection{Factors influencing Neurofeedback}

Description and analysis of different types of \gls{nfb} implementation was subject to several studies \citep{Arns2014,
  Enriquez2017, Vernon2004, Jeunet2018} but to our knowledge none used statistical tools to quantify their influence on
clinical endpoints. 

The interpretation of results provided by this analysis requires some care granted that each method offer slightly
differen results. These discrepancies can easily be explained by the varying hypothesis of each models and actually
offered interesting insight into the results and their significance. For instance, the decision tree method is a
non-linear  method that accounts for variables interaction unlike the other two. Moreover, the decision tree  is more unstable \citep{dwyer2007}, meaning that a small change in the
data could cause an important change in the structure of the optimal decision tree.\todo{why is that, why does this
  matter? detail briefly or remove}

Surprisingly, the number of sessions was not found as a significant factor by any method, which was somewhat
in contradiction with existing literature. For instance, \citet{Enriquez2017} insisted on the fact that the number of
sessions should be chosen carefully to avoid "overtraining". Moreover, \citet{Arns2014} stated that performing less than
20 \gls{nfb} sessions led to smaller effects. Similarly, \citet{Vernon2004} observed that positive changes in the \gls{eeg}
and behavioral performance occurred after a minimum of 20 sessions. Interestingly, this last author also suggests that

the location of the \gls{nfb} training may also be an important contributing factor to clinical effectiveness.
This was however recently investigated by a recent study \citep{Minder2018} showing that
performing \gls{nfb} at school or at the clinic has no significant
impact on treatment response. 
\todo{you jump from location back to number of sessions. Perhaps cite minders twice and separate the points more
  clearly. }

The fact that the number of sessions was not identified as a positively contributing factor, might be explain by the
presence of only one data point with 20 sessions or less. Possibly, the temporal threshold of efficacy was passed for
all included studies making the identification of this factor unlikely on this dataset. However, regardless of its
statistical significance, the coefficient found by the \gls{wls} was negative, meaning that as expected,
the more sessions performed the more efficient the \gls{nfb} seemed to be. 

The type of \gls{nfb} protocol was not identified by all the three methods but it seemed to influence the \gls{nfb} results
according to 2 methods.  In particular, the theta down protocol appeared more efficient and the \gls{smr} protocol
associated lower \gls{es}. This importance granted by the methods to the \gls{nfb} protocols was somewhat lower to
expectations given their centrality in the neurophysiological mode of action and subsequent expected impact on
therapeutci effectiveness \citet{Vernon2004}.  A possible explanation for this result is that these
protocols were equally efficacious to the populations they were offered to and thereby did not constitute a significant
explanatory factor. This result, however, does not preclude a combined and personalized strategy (offer the right
protocol to the right kid) to further improve performance as previously by \citet{Alkoby2017}.

Several factors were selected by all three methods with the same direction of influence: if the rater was
probably blind to the treatment, the treatment length, and the EEG quality.\todo{list in the same order you then use to
discuss below} 

First, this analysis pointed out the fact that recording \gls{eeg} in good conditions seemed to lead to better results,
which could be explained by the fact that better signal quality enabled to extract more correctly the \gls{eeg} patterns
linked to \gls{adhd} and henceforth led to better learning and therapeutic efficacy. However, it remained difficult to
really assess the quality of the hardware because little information was provided in the studies.  

Next, it appeared here that the longer the treatment the less efficient it became. Arguably, the treatment length was a
proxy for treatment intensity, which meant that a treatment that was short in length (and consequently intense in pace)
was more likely to succeed. This hypothesis was back-up by the fact that the variable \emph{session pace} (number of
sessions per week) was also associated with larger \gls{es} according to the \gls{wls} and \gls{lasso}. Impact of the
intensity of treatment have been investigated by \citep{Rogala2016} on healthy adults: it was observed that studies with
at least four training sessions completed on consecutive days were all successful. 

As expected, the assessment of symptoms by non-blind raters led to far more favorable results than by \gls{pblind} raters -
result widely expected and in close compliance with existing meta-analysis \citep{Cortese2016, Micoulaud2014}. 

Strangely, the data provided did not exactly matched the widely accpeted hypothesis stating that the difference between
\gls{mprox} and \gls{pblind} can solely be explained by the placebo effect. The results provided by the decision tree
illustrated by Figure~\ref{Figure:factors_analysis_decision_tree_results} provide a good insight to comment on this.
Indeed, the top node splits on one hand 43 observations corresponding to \gls{mprox} raters and, on the other hand, 19 observations corresponding to \gls{pblind}. If the differences observed between \gls{pblind} and \gls{mprox} raters were due to the
placebo effect, one would expect to find in the \gls{mprox} part of the decision tree factors linked to the perception
of or implication into the treatment. It was, indeed the case: session and treatment length were found but with a
opposite contribution than expected if they were contributing to a placebo effect.  Indeed, one would expect that the
longer the session and the treatment, the higher the placebo effect and the larger the within-\gls{es}.
Yet, the opposite was found, somewhat invalidating the hypothesis. 

Another way to highlight a possible placebo effect, was to focus on the difference between \gls{pblind} and \gls{mprox}
raters at pre- and post-test. The expected differences of ratings between teachers and parents have been extensively
studied \citep{Sollie2013, Narad2015, Minder2018} noting that teachers were more likely to underrate a child's severity,
and especially so for younger children. This is in good line with results of studies included in this work as illustrated in~\ref{Figure:discussion_on_placebo_effect_colors_2-columns_fitting_image}.  
Besides, it is also clear that there is more variability in
teachers' scores compared to parents', which could partly explained the lower \gls{es} obtained for \gls{pblind} raters
since the variability will influence the denominator. 
These results altogether suggest that \gls{pblind} assessments could hardly be used to assess placebo effect as they seemed
to be blinder to symptoms than to intervention. 
In the absence of ethically \cite{} and technically \cite{}\todo{look at FDA pre-subs slides to find reference about the
  two previous points} feasible sham for \gls{nfb} protocols \citep{World-Medical-Association2000}, it
is necessary to fallback on acceptable methodological alternative to the demonstration of clinical effectiveness. Among
those are analysis of neuromarkers collected during \gls{nfb} treatment demonstrating that patients do \emph{control} the
trained neuromarkers; that they \emph{learn} (reinforce control over time), and that these possibly lead to lasting brain
reorganisation (e.g. changes in their baseline resting state activity). The specifity of these changes, in relation to,
which neuromarkers were trained and clinical improvement will be an essential component of this demonstration.  

%It would have been interesting to study the influence of some other factors such as the delay between brain state and feedback signal as well as the type of \gls{nfb} metaphor and feedback used as pointed out by \citet{Alkoby2017}, but these information were rarely available in studies. Besides, to add more reliability to these results it should be preferable to add more studies, particularly studies with teachers assessments (considered as \gls{pblind}). 

% words number = 1518
