% adding the line below for Multifile document support with LatexTools Sublime package 
%!TEX root = manuscript.tex

% conclusion

\section{Conclusion}

In this work we provide additional elements in favor of the effectiveness of \gls{nfb} for the treatment of \gls{adhd}. First, 
we confirm that a subgroup of standard \gls{nfb} studies shows a statistically significant improvement on \gls{pblind} 
assessments (k = 4 studies instead of 3, n = 283 patients instead of 158 \citet{Cortese2016}). 

Second, we identify technical factors as positive contributors to clinical effectiveness, which strongly suggests 
that it is mediated by a real mechanism of action based on \gls{eeg} conditioning. Equally, treatment intensity was also found to 
contribute, corroborating what is known from learning theory (memory consolidation) \citep{Mowrer1960}; that is to say, 
a more intense treatment leads to an increased clinical efficacy.

While these findings certainly contribute to the debate, this work also suggests that the ultimate demonstration of evidence 
remains out of reach, as teachers’ assessments were partly invalidated as a proxy for the quantification of the placebo effect. 
As a consequence, using \gls{pblind} endpoints to address the specificity of the clinical efficacy is not recommended 
and we instead advise a reliance on other available methodological tools. These tools include sham \gls{nfb} and neuromarker 
analysis investigating the specificity of the \gls{eeg} changes with respect to trained neuromarkers as well as changes 
in clinical endpoints.

This work also offers an open-source toolbox for running meta-analysis and \gls{saob}: the code and data used are available, 
thus ensuring the transparency and replicability of these analysis, as well as fostering future ones.
Regarding perspectives, this two-fold methodological framework applied to \gls{nfb} for \gls{adhd} could be applied for other \gls{nfb} applications.

% 240
