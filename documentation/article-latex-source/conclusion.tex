% adding the line below for Multifile document support with LatexTools Sublime package 
%!TEX root = manuscript.tex

% conclusion

\section{Conclusion}

%This work confirms \citet{Cortese2016}'s findings in the light of recently published clinical works.
%In particular, studies following a standard protocol as defined by \citet{Arns2014} show significant 
%clinical improvements on \gls{pblind} raters (k = 4 studies instead of 3 \citet{Cortese2016}).

%Besides a meta-analysis, a new method is suggested here to tackle the high heterogeneity of clinical data 
%available on \gls{nfb}. This method aims at identifying factors that are positively or negatively contributing 
%to \gls{nfb} efficacy. Three factors were consistently found to explain clinical within-\gls{es}. First, the 
%quality of acquisition of the \gls{eeg} was positively correlated with clinical efficacy. This supports a mode of 
%action through specific \gls{eeg} training. Likewise, treatment intensity was found to contribute, corroborating what 
%is known from learning theory (memory consolidation) \citep{Mowrer1960}, that is, a more intense treatment leads to 
%an increased clinical efficacy. Finally, results show that the therapeutic efficacy measured by teachers is reduced 
%compared to that measured by parents. This result has long been documented and it is widely advanced that this 
%difference is solely imputable to the amplitude of placebo effect in \gls{nfb}. However, the data presented in this article, 
%in line with the most recent work on the topic \citep{Sollie2013, Narad2015, Minder2018} 
%tends to refute this hypothesis and suggests instead that teachers are simply less likely to be exposed to symptoms. 
%As a consequence, using \gls{pblind} endpoints to address the specificity of the clinical efficacy is not recommended 
%and we advice instead to rely on other available methodological tools. \textcolor{red}{Those include sham \gls{nfb} and neuromarker 
%analysis investigating the specificity of the \gls{eeg} changes with respect to trained neuromarkers as well as changes 
%in clinical endpoints.}
%
%These elements converge to the conclusion that existing methodologies, in particular the double blind design, 
%traditionally used to assess pharmacological treatments in neuropsychiatric conditions may not be fully fitted to the evaluation 
%of medical devices. The series of results presented here, however, suggest the presence of a genuine signal in favor of 
%the therapeutic efficacy of \gls{nfb}. A signal that should nonetheless be studied further using the aforementioned methodological 
%tools, neuromarker analysis in the first place.


\textcolor{red}{In this work we bring additional elements in favor of the effectiveness of \gls{nfb} for the treatment of \gls{adhd}. First, 
we confirm that a subgroup of standard \gls{nfb} studies show statistically significant improvement on \gls{pblind} 
assessments (k = 4 studies instead of 3, n = 283 patients instead of 158} \citet{Cortese2016}). 

\textcolor{red}{Second, we identify technical factors as positive contributors to clinical effectiveness, which strongly suggests 
that it is mediated by a real mechanism of action based on \gls{eeg} conditioning. Likewise, treatment intensity was found to 
contribute too, corroborating what is known from learning theory (memory consolidation)} \citep{Mowrer1960}\textcolor{red}{, that is, a more intense treatment leads to 
an increased clinical efficacy.}

\textcolor{red}{While these findings certainly contribute to the debate, this work also suggests that the ultimate demonstration of evidence 
remains (out of reach) as teachers’ assessments were partly invalidated as a proxy for the quantification of the placebo effect. 
As a consequence, using \gls{pblind} endpoints to address the specificity of the clinical efficacy is not recommended 
and we advice instead to rely on other available methodological tools. Those include sham \gls{nfb} and neuromarker 
analysis investigating the specificity of the \gls{eeg} changes with respect to trained neuromarkers as well as changes 
in clinical endpoints.}

\textcolor{red}{This work also offers an open-source toolbox for running meta-analysis and \gls{saob}: the code and data used are available 
assuring the transparency and replicability of these analysis, as well as fostering future ones.}

% 375
