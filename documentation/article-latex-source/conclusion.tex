% adding the line below for Multifile document support with LatexTools Sublime package 
%!TEX root = manuscript.tex

% conclusion

\section{Conclusion}


This work confirms \citet{Cortese2016}'s findings in the light of recent published clinical work.
In particular, studies following a standard protocol as defined by \citet{Arns2014} show significant 
clinical improvements on \gls{pblind} raters (k = 4 studies instead of 3 \citet{Cortese2016}).

Besides this meta-analysis, a new method is suggested to tackle the high heterogeneity of clinical data 
available on \gls{nfb}. This method aims at identifying factors as positively or negatively contributing 
to \gls{nfb} efficacy. Three factors were consistently found to statistically significantly explain clinical 
within-\gls{es}. First, the quality of acquisition of the \gls{eeg} was positively correlated with clinical 
efficacy. It confirms a mode of action through specific \gls{eeg} training. Likewise, treatment intensity was 
found to contribute, corroborating what is known of learning theory (memory consolidation) \citep{Mowrer1960}, 
that is a more intense treatment leads to an increased clinical efficacy. Finally, results show that the therapeutic 
efficacy measured by teachers is reduced compared to that measured by parents. This result has long been documented 
and it is widely accepted that this difference is solely imputable to the amplitude of placebo effect in \gls{nfb}. 
However, the data presented in this article tend to refute this hypothesis and suggest that teachers are simply 
exposed to less symptoms, in line with the most recent work on the topic. As a consequence, using \gls{pblind} 
endpoints to address the specificity of the clinical efficacy is not recommended and the community should instead 
rely on other available methodological tools. Those include sham \gls{nfb} and neuromarker analysis investigating 
the specificity of the \gls{eeg} changes with respect to trained neuromarkers and changes in clinical endpoints.

These elements converge to the conclusion that existing methodologies traditionally used to assess drug treatments 
in neuropsychiatric conditions may not be fully fit to the evaluation of medical devices. The series of results presented 
here however suggest the presence of a genuine signal in favor of the therapeutic efficacy of \gls{nfb}. A signal that 
should nonetheless be strengthen using the aforementioned methodological tools, neuromarker analysis in the first place.

This work also offers an open source tool for running meta-analysis and \gls{saob}: the code and data used are available, 
assuring the transparency and replicability of these analysis. 

% 241
